\documentclass[11pt]{article}

    \usepackage[breakable]{tcolorbox}
    \usepackage{parskip} % Stop auto-indenting (to mimic markdown behaviour)
    
    \usepackage{iftex}
    \ifPDFTeX
    	\usepackage[T1]{fontenc}
    	\usepackage{mathpazo}
    \else
    	\usepackage{fontspec}
    \fi

    % Basic figure setup, for now with no caption control since it's done
    % automatically by Pandoc (which extracts ![](path) syntax from Markdown).
    \usepackage{graphicx}
    % Maintain compatibility with old templates. Remove in nbconvert 6.0
    \let\Oldincludegraphics\includegraphics
    % Ensure that by default, figures have no caption (until we provide a
    % proper Figure object with a Caption API and a way to capture that
    % in the conversion process - todo).
    \usepackage{caption}
    \DeclareCaptionFormat{nocaption}{}
    \captionsetup{format=nocaption,aboveskip=0pt,belowskip=0pt}

    \usepackage[Export]{adjustbox} % Used to constrain images to a maximum size
    \adjustboxset{max size={0.9\linewidth}{0.9\paperheight}}
    \usepackage{float}
    \floatplacement{figure}{H} % forces figures to be placed at the correct location
    \usepackage{xcolor} % Allow colors to be defined
    \usepackage{enumerate} % Needed for markdown enumerations to work
    \usepackage{geometry} % Used to adjust the document margins
    \usepackage{amsmath} % Equations
    \usepackage{amssymb} % Equations
    \usepackage{textcomp} % defines textquotesingle
    % Hack from http://tex.stackexchange.com/a/47451/13684:
    \AtBeginDocument{%
        \def\PYZsq{\textquotesingle}% Upright quotes in Pygmentized code
    }
    \usepackage{upquote} % Upright quotes for verbatim code
    \usepackage{eurosym} % defines \euro
    \usepackage[mathletters]{ucs} % Extended unicode (utf-8) support
    \usepackage{fancyvrb} % verbatim replacement that allows latex
    \usepackage{grffile} % extends the file name processing of package graphics 
                         % to support a larger range
    \makeatletter % fix for grffile with XeLaTeX
    \def\Gread@@xetex#1{%
      \IfFileExists{"\Gin@base".bb}%
      {\Gread@eps{\Gin@base.bb}}%
      {\Gread@@xetex@aux#1}%
    }
    \makeatother

    % The hyperref package gives us a pdf with properly built
    % internal navigation ('pdf bookmarks' for the table of contents,
    % internal cross-reference links, web links for URLs, etc.)
    \usepackage{hyperref}
    % The default LaTeX title has an obnoxious amount of whitespace. By default,
    % titling removes some of it. It also provides customization options.
    \usepackage{titling}
    \usepackage{longtable} % longtable support required by pandoc >1.10
    \usepackage{booktabs}  % table support for pandoc > 1.12.2
    \usepackage[inline]{enumitem} % IRkernel/repr support (it uses the enumerate* environment)
    \usepackage[normalem]{ulem} % ulem is needed to support strikethroughs (\sout)
                                % normalem makes italics be italics, not underlines
    \usepackage{mathrsfs}
    

    
    % Colors for the hyperref package
    \definecolor{urlcolor}{rgb}{0,.145,.698}
    \definecolor{linkcolor}{rgb}{.71,0.21,0.01}
    \definecolor{citecolor}{rgb}{.12,.54,.11}

    % ANSI colors
    \definecolor{ansi-black}{HTML}{3E424D}
    \definecolor{ansi-black-intense}{HTML}{282C36}
    \definecolor{ansi-red}{HTML}{E75C58}
    \definecolor{ansi-red-intense}{HTML}{B22B31}
    \definecolor{ansi-green}{HTML}{00A250}
    \definecolor{ansi-green-intense}{HTML}{007427}
    \definecolor{ansi-yellow}{HTML}{DDB62B}
    \definecolor{ansi-yellow-intense}{HTML}{B27D12}
    \definecolor{ansi-blue}{HTML}{208FFB}
    \definecolor{ansi-blue-intense}{HTML}{0065CA}
    \definecolor{ansi-magenta}{HTML}{D160C4}
    \definecolor{ansi-magenta-intense}{HTML}{A03196}
    \definecolor{ansi-cyan}{HTML}{60C6C8}
    \definecolor{ansi-cyan-intense}{HTML}{258F8F}
    \definecolor{ansi-white}{HTML}{C5C1B4}
    \definecolor{ansi-white-intense}{HTML}{A1A6B2}
    \definecolor{ansi-default-inverse-fg}{HTML}{FFFFFF}
    \definecolor{ansi-default-inverse-bg}{HTML}{000000}

    % commands and environments needed by pandoc snippets
    % extracted from the output of `pandoc -s`
    \providecommand{\tightlist}{%
      \setlength{\itemsep}{0pt}\setlength{\parskip}{0pt}}
    \DefineVerbatimEnvironment{Highlighting}{Verbatim}{commandchars=\\\{\}}
    % Add ',fontsize=\small' for more characters per line
    \newenvironment{Shaded}{}{}
    \newcommand{\KeywordTok}[1]{\textcolor[rgb]{0.00,0.44,0.13}{\textbf{{#1}}}}
    \newcommand{\DataTypeTok}[1]{\textcolor[rgb]{0.56,0.13,0.00}{{#1}}}
    \newcommand{\DecValTok}[1]{\textcolor[rgb]{0.25,0.63,0.44}{{#1}}}
    \newcommand{\BaseNTok}[1]{\textcolor[rgb]{0.25,0.63,0.44}{{#1}}}
    \newcommand{\FloatTok}[1]{\textcolor[rgb]{0.25,0.63,0.44}{{#1}}}
    \newcommand{\CharTok}[1]{\textcolor[rgb]{0.25,0.44,0.63}{{#1}}}
    \newcommand{\StringTok}[1]{\textcolor[rgb]{0.25,0.44,0.63}{{#1}}}
    \newcommand{\CommentTok}[1]{\textcolor[rgb]{0.38,0.63,0.69}{\textit{{#1}}}}
    \newcommand{\OtherTok}[1]{\textcolor[rgb]{0.00,0.44,0.13}{{#1}}}
    \newcommand{\AlertTok}[1]{\textcolor[rgb]{1.00,0.00,0.00}{\textbf{{#1}}}}
    \newcommand{\FunctionTok}[1]{\textcolor[rgb]{0.02,0.16,0.49}{{#1}}}
    \newcommand{\RegionMarkerTok}[1]{{#1}}
    \newcommand{\ErrorTok}[1]{\textcolor[rgb]{1.00,0.00,0.00}{\textbf{{#1}}}}
    \newcommand{\NormalTok}[1]{{#1}}
    
    % Additional commands for more recent versions of Pandoc
    \newcommand{\ConstantTok}[1]{\textcolor[rgb]{0.53,0.00,0.00}{{#1}}}
    \newcommand{\SpecialCharTok}[1]{\textcolor[rgb]{0.25,0.44,0.63}{{#1}}}
    \newcommand{\VerbatimStringTok}[1]{\textcolor[rgb]{0.25,0.44,0.63}{{#1}}}
    \newcommand{\SpecialStringTok}[1]{\textcolor[rgb]{0.73,0.40,0.53}{{#1}}}
    \newcommand{\ImportTok}[1]{{#1}}
    \newcommand{\DocumentationTok}[1]{\textcolor[rgb]{0.73,0.13,0.13}{\textit{{#1}}}}
    \newcommand{\AnnotationTok}[1]{\textcolor[rgb]{0.38,0.63,0.69}{\textbf{\textit{{#1}}}}}
    \newcommand{\CommentVarTok}[1]{\textcolor[rgb]{0.38,0.63,0.69}{\textbf{\textit{{#1}}}}}
    \newcommand{\VariableTok}[1]{\textcolor[rgb]{0.10,0.09,0.49}{{#1}}}
    \newcommand{\ControlFlowTok}[1]{\textcolor[rgb]{0.00,0.44,0.13}{\textbf{{#1}}}}
    \newcommand{\OperatorTok}[1]{\textcolor[rgb]{0.40,0.40,0.40}{{#1}}}
    \newcommand{\BuiltInTok}[1]{{#1}}
    \newcommand{\ExtensionTok}[1]{{#1}}
    \newcommand{\PreprocessorTok}[1]{\textcolor[rgb]{0.74,0.48,0.00}{{#1}}}
    \newcommand{\AttributeTok}[1]{\textcolor[rgb]{0.49,0.56,0.16}{{#1}}}
    \newcommand{\InformationTok}[1]{\textcolor[rgb]{0.38,0.63,0.69}{\textbf{\textit{{#1}}}}}
    \newcommand{\WarningTok}[1]{\textcolor[rgb]{0.38,0.63,0.69}{\textbf{\textit{{#1}}}}}
    
    
    % Define a nice break command that doesn't care if a line doesn't already
    % exist.
    \def\br{\hspace*{\fill} \\* }
    % Math Jax compatibility definitions
    \def\gt{>}
    \def\lt{<}
    \let\Oldtex\TeX
    \let\Oldlatex\LaTeX
    \renewcommand{\TeX}{\textrm{\Oldtex}}
    \renewcommand{\LaTeX}{\textrm{\Oldlatex}}
    % Document parameters
    % Document title
    \title{ISyE6420\_final}
    
    
    
    
    
% Pygments definitions
\makeatletter
\def\PY@reset{\let\PY@it=\relax \let\PY@bf=\relax%
    \let\PY@ul=\relax \let\PY@tc=\relax%
    \let\PY@bc=\relax \let\PY@ff=\relax}
\def\PY@tok#1{\csname PY@tok@#1\endcsname}
\def\PY@toks#1+{\ifx\relax#1\empty\else%
    \PY@tok{#1}\expandafter\PY@toks\fi}
\def\PY@do#1{\PY@bc{\PY@tc{\PY@ul{%
    \PY@it{\PY@bf{\PY@ff{#1}}}}}}}
\def\PY#1#2{\PY@reset\PY@toks#1+\relax+\PY@do{#2}}

\expandafter\def\csname PY@tok@w\endcsname{\def\PY@tc##1{\textcolor[rgb]{0.73,0.73,0.73}{##1}}}
\expandafter\def\csname PY@tok@c\endcsname{\let\PY@it=\textit\def\PY@tc##1{\textcolor[rgb]{0.25,0.50,0.50}{##1}}}
\expandafter\def\csname PY@tok@cp\endcsname{\def\PY@tc##1{\textcolor[rgb]{0.74,0.48,0.00}{##1}}}
\expandafter\def\csname PY@tok@k\endcsname{\let\PY@bf=\textbf\def\PY@tc##1{\textcolor[rgb]{0.00,0.50,0.00}{##1}}}
\expandafter\def\csname PY@tok@kp\endcsname{\def\PY@tc##1{\textcolor[rgb]{0.00,0.50,0.00}{##1}}}
\expandafter\def\csname PY@tok@kt\endcsname{\def\PY@tc##1{\textcolor[rgb]{0.69,0.00,0.25}{##1}}}
\expandafter\def\csname PY@tok@o\endcsname{\def\PY@tc##1{\textcolor[rgb]{0.40,0.40,0.40}{##1}}}
\expandafter\def\csname PY@tok@ow\endcsname{\let\PY@bf=\textbf\def\PY@tc##1{\textcolor[rgb]{0.67,0.13,1.00}{##1}}}
\expandafter\def\csname PY@tok@nb\endcsname{\def\PY@tc##1{\textcolor[rgb]{0.00,0.50,0.00}{##1}}}
\expandafter\def\csname PY@tok@nf\endcsname{\def\PY@tc##1{\textcolor[rgb]{0.00,0.00,1.00}{##1}}}
\expandafter\def\csname PY@tok@nc\endcsname{\let\PY@bf=\textbf\def\PY@tc##1{\textcolor[rgb]{0.00,0.00,1.00}{##1}}}
\expandafter\def\csname PY@tok@nn\endcsname{\let\PY@bf=\textbf\def\PY@tc##1{\textcolor[rgb]{0.00,0.00,1.00}{##1}}}
\expandafter\def\csname PY@tok@ne\endcsname{\let\PY@bf=\textbf\def\PY@tc##1{\textcolor[rgb]{0.82,0.25,0.23}{##1}}}
\expandafter\def\csname PY@tok@nv\endcsname{\def\PY@tc##1{\textcolor[rgb]{0.10,0.09,0.49}{##1}}}
\expandafter\def\csname PY@tok@no\endcsname{\def\PY@tc##1{\textcolor[rgb]{0.53,0.00,0.00}{##1}}}
\expandafter\def\csname PY@tok@nl\endcsname{\def\PY@tc##1{\textcolor[rgb]{0.63,0.63,0.00}{##1}}}
\expandafter\def\csname PY@tok@ni\endcsname{\let\PY@bf=\textbf\def\PY@tc##1{\textcolor[rgb]{0.60,0.60,0.60}{##1}}}
\expandafter\def\csname PY@tok@na\endcsname{\def\PY@tc##1{\textcolor[rgb]{0.49,0.56,0.16}{##1}}}
\expandafter\def\csname PY@tok@nt\endcsname{\let\PY@bf=\textbf\def\PY@tc##1{\textcolor[rgb]{0.00,0.50,0.00}{##1}}}
\expandafter\def\csname PY@tok@nd\endcsname{\def\PY@tc##1{\textcolor[rgb]{0.67,0.13,1.00}{##1}}}
\expandafter\def\csname PY@tok@s\endcsname{\def\PY@tc##1{\textcolor[rgb]{0.73,0.13,0.13}{##1}}}
\expandafter\def\csname PY@tok@sd\endcsname{\let\PY@it=\textit\def\PY@tc##1{\textcolor[rgb]{0.73,0.13,0.13}{##1}}}
\expandafter\def\csname PY@tok@si\endcsname{\let\PY@bf=\textbf\def\PY@tc##1{\textcolor[rgb]{0.73,0.40,0.53}{##1}}}
\expandafter\def\csname PY@tok@se\endcsname{\let\PY@bf=\textbf\def\PY@tc##1{\textcolor[rgb]{0.73,0.40,0.13}{##1}}}
\expandafter\def\csname PY@tok@sr\endcsname{\def\PY@tc##1{\textcolor[rgb]{0.73,0.40,0.53}{##1}}}
\expandafter\def\csname PY@tok@ss\endcsname{\def\PY@tc##1{\textcolor[rgb]{0.10,0.09,0.49}{##1}}}
\expandafter\def\csname PY@tok@sx\endcsname{\def\PY@tc##1{\textcolor[rgb]{0.00,0.50,0.00}{##1}}}
\expandafter\def\csname PY@tok@m\endcsname{\def\PY@tc##1{\textcolor[rgb]{0.40,0.40,0.40}{##1}}}
\expandafter\def\csname PY@tok@gh\endcsname{\let\PY@bf=\textbf\def\PY@tc##1{\textcolor[rgb]{0.00,0.00,0.50}{##1}}}
\expandafter\def\csname PY@tok@gu\endcsname{\let\PY@bf=\textbf\def\PY@tc##1{\textcolor[rgb]{0.50,0.00,0.50}{##1}}}
\expandafter\def\csname PY@tok@gd\endcsname{\def\PY@tc##1{\textcolor[rgb]{0.63,0.00,0.00}{##1}}}
\expandafter\def\csname PY@tok@gi\endcsname{\def\PY@tc##1{\textcolor[rgb]{0.00,0.63,0.00}{##1}}}
\expandafter\def\csname PY@tok@gr\endcsname{\def\PY@tc##1{\textcolor[rgb]{1.00,0.00,0.00}{##1}}}
\expandafter\def\csname PY@tok@ge\endcsname{\let\PY@it=\textit}
\expandafter\def\csname PY@tok@gs\endcsname{\let\PY@bf=\textbf}
\expandafter\def\csname PY@tok@gp\endcsname{\let\PY@bf=\textbf\def\PY@tc##1{\textcolor[rgb]{0.00,0.00,0.50}{##1}}}
\expandafter\def\csname PY@tok@go\endcsname{\def\PY@tc##1{\textcolor[rgb]{0.53,0.53,0.53}{##1}}}
\expandafter\def\csname PY@tok@gt\endcsname{\def\PY@tc##1{\textcolor[rgb]{0.00,0.27,0.87}{##1}}}
\expandafter\def\csname PY@tok@err\endcsname{\def\PY@bc##1{\setlength{\fboxsep}{0pt}\fcolorbox[rgb]{1.00,0.00,0.00}{1,1,1}{\strut ##1}}}
\expandafter\def\csname PY@tok@kc\endcsname{\let\PY@bf=\textbf\def\PY@tc##1{\textcolor[rgb]{0.00,0.50,0.00}{##1}}}
\expandafter\def\csname PY@tok@kd\endcsname{\let\PY@bf=\textbf\def\PY@tc##1{\textcolor[rgb]{0.00,0.50,0.00}{##1}}}
\expandafter\def\csname PY@tok@kn\endcsname{\let\PY@bf=\textbf\def\PY@tc##1{\textcolor[rgb]{0.00,0.50,0.00}{##1}}}
\expandafter\def\csname PY@tok@kr\endcsname{\let\PY@bf=\textbf\def\PY@tc##1{\textcolor[rgb]{0.00,0.50,0.00}{##1}}}
\expandafter\def\csname PY@tok@bp\endcsname{\def\PY@tc##1{\textcolor[rgb]{0.00,0.50,0.00}{##1}}}
\expandafter\def\csname PY@tok@fm\endcsname{\def\PY@tc##1{\textcolor[rgb]{0.00,0.00,1.00}{##1}}}
\expandafter\def\csname PY@tok@vc\endcsname{\def\PY@tc##1{\textcolor[rgb]{0.10,0.09,0.49}{##1}}}
\expandafter\def\csname PY@tok@vg\endcsname{\def\PY@tc##1{\textcolor[rgb]{0.10,0.09,0.49}{##1}}}
\expandafter\def\csname PY@tok@vi\endcsname{\def\PY@tc##1{\textcolor[rgb]{0.10,0.09,0.49}{##1}}}
\expandafter\def\csname PY@tok@vm\endcsname{\def\PY@tc##1{\textcolor[rgb]{0.10,0.09,0.49}{##1}}}
\expandafter\def\csname PY@tok@sa\endcsname{\def\PY@tc##1{\textcolor[rgb]{0.73,0.13,0.13}{##1}}}
\expandafter\def\csname PY@tok@sb\endcsname{\def\PY@tc##1{\textcolor[rgb]{0.73,0.13,0.13}{##1}}}
\expandafter\def\csname PY@tok@sc\endcsname{\def\PY@tc##1{\textcolor[rgb]{0.73,0.13,0.13}{##1}}}
\expandafter\def\csname PY@tok@dl\endcsname{\def\PY@tc##1{\textcolor[rgb]{0.73,0.13,0.13}{##1}}}
\expandafter\def\csname PY@tok@s2\endcsname{\def\PY@tc##1{\textcolor[rgb]{0.73,0.13,0.13}{##1}}}
\expandafter\def\csname PY@tok@sh\endcsname{\def\PY@tc##1{\textcolor[rgb]{0.73,0.13,0.13}{##1}}}
\expandafter\def\csname PY@tok@s1\endcsname{\def\PY@tc##1{\textcolor[rgb]{0.73,0.13,0.13}{##1}}}
\expandafter\def\csname PY@tok@mb\endcsname{\def\PY@tc##1{\textcolor[rgb]{0.40,0.40,0.40}{##1}}}
\expandafter\def\csname PY@tok@mf\endcsname{\def\PY@tc##1{\textcolor[rgb]{0.40,0.40,0.40}{##1}}}
\expandafter\def\csname PY@tok@mh\endcsname{\def\PY@tc##1{\textcolor[rgb]{0.40,0.40,0.40}{##1}}}
\expandafter\def\csname PY@tok@mi\endcsname{\def\PY@tc##1{\textcolor[rgb]{0.40,0.40,0.40}{##1}}}
\expandafter\def\csname PY@tok@il\endcsname{\def\PY@tc##1{\textcolor[rgb]{0.40,0.40,0.40}{##1}}}
\expandafter\def\csname PY@tok@mo\endcsname{\def\PY@tc##1{\textcolor[rgb]{0.40,0.40,0.40}{##1}}}
\expandafter\def\csname PY@tok@ch\endcsname{\let\PY@it=\textit\def\PY@tc##1{\textcolor[rgb]{0.25,0.50,0.50}{##1}}}
\expandafter\def\csname PY@tok@cm\endcsname{\let\PY@it=\textit\def\PY@tc##1{\textcolor[rgb]{0.25,0.50,0.50}{##1}}}
\expandafter\def\csname PY@tok@cpf\endcsname{\let\PY@it=\textit\def\PY@tc##1{\textcolor[rgb]{0.25,0.50,0.50}{##1}}}
\expandafter\def\csname PY@tok@c1\endcsname{\let\PY@it=\textit\def\PY@tc##1{\textcolor[rgb]{0.25,0.50,0.50}{##1}}}
\expandafter\def\csname PY@tok@cs\endcsname{\let\PY@it=\textit\def\PY@tc##1{\textcolor[rgb]{0.25,0.50,0.50}{##1}}}

\def\PYZbs{\char`\\}
\def\PYZus{\char`\_}
\def\PYZob{\char`\{}
\def\PYZcb{\char`\}}
\def\PYZca{\char`\^}
\def\PYZam{\char`\&}
\def\PYZlt{\char`\<}
\def\PYZgt{\char`\>}
\def\PYZsh{\char`\#}
\def\PYZpc{\char`\%}
\def\PYZdl{\char`\$}
\def\PYZhy{\char`\-}
\def\PYZsq{\char`\'}
\def\PYZdq{\char`\"}
\def\PYZti{\char`\~}
% for compatibility with earlier versions
\def\PYZat{@}
\def\PYZlb{[}
\def\PYZrb{]}
\makeatother


    % For linebreaks inside Verbatim environment from package fancyvrb. 
    \makeatletter
        \newbox\Wrappedcontinuationbox 
        \newbox\Wrappedvisiblespacebox 
        \newcommand*\Wrappedvisiblespace {\textcolor{red}{\textvisiblespace}} 
        \newcommand*\Wrappedcontinuationsymbol {\textcolor{red}{\llap{\tiny$\m@th\hookrightarrow$}}} 
        \newcommand*\Wrappedcontinuationindent {3ex } 
        \newcommand*\Wrappedafterbreak {\kern\Wrappedcontinuationindent\copy\Wrappedcontinuationbox} 
        % Take advantage of the already applied Pygments mark-up to insert 
        % potential linebreaks for TeX processing. 
        %        {, <, #, %, $, ' and ": go to next line. 
        %        _, }, ^, &, >, - and ~: stay at end of broken line. 
        % Use of \textquotesingle for straight quote. 
        \newcommand*\Wrappedbreaksatspecials {% 
            \def\PYGZus{\discretionary{\char`\_}{\Wrappedafterbreak}{\char`\_}}% 
            \def\PYGZob{\discretionary{}{\Wrappedafterbreak\char`\{}{\char`\{}}% 
            \def\PYGZcb{\discretionary{\char`\}}{\Wrappedafterbreak}{\char`\}}}% 
            \def\PYGZca{\discretionary{\char`\^}{\Wrappedafterbreak}{\char`\^}}% 
            \def\PYGZam{\discretionary{\char`\&}{\Wrappedafterbreak}{\char`\&}}% 
            \def\PYGZlt{\discretionary{}{\Wrappedafterbreak\char`\<}{\char`\<}}% 
            \def\PYGZgt{\discretionary{\char`\>}{\Wrappedafterbreak}{\char`\>}}% 
            \def\PYGZsh{\discretionary{}{\Wrappedafterbreak\char`\#}{\char`\#}}% 
            \def\PYGZpc{\discretionary{}{\Wrappedafterbreak\char`\%}{\char`\%}}% 
            \def\PYGZdl{\discretionary{}{\Wrappedafterbreak\char`\$}{\char`\$}}% 
            \def\PYGZhy{\discretionary{\char`\-}{\Wrappedafterbreak}{\char`\-}}% 
            \def\PYGZsq{\discretionary{}{\Wrappedafterbreak\textquotesingle}{\textquotesingle}}% 
            \def\PYGZdq{\discretionary{}{\Wrappedafterbreak\char`\"}{\char`\"}}% 
            \def\PYGZti{\discretionary{\char`\~}{\Wrappedafterbreak}{\char`\~}}% 
        } 
        % Some characters . , ; ? ! / are not pygmentized. 
        % This macro makes them "active" and they will insert potential linebreaks 
        \newcommand*\Wrappedbreaksatpunct {% 
            \lccode`\~`\.\lowercase{\def~}{\discretionary{\hbox{\char`\.}}{\Wrappedafterbreak}{\hbox{\char`\.}}}% 
            \lccode`\~`\,\lowercase{\def~}{\discretionary{\hbox{\char`\,}}{\Wrappedafterbreak}{\hbox{\char`\,}}}% 
            \lccode`\~`\;\lowercase{\def~}{\discretionary{\hbox{\char`\;}}{\Wrappedafterbreak}{\hbox{\char`\;}}}% 
            \lccode`\~`\:\lowercase{\def~}{\discretionary{\hbox{\char`\:}}{\Wrappedafterbreak}{\hbox{\char`\:}}}% 
            \lccode`\~`\?\lowercase{\def~}{\discretionary{\hbox{\char`\?}}{\Wrappedafterbreak}{\hbox{\char`\?}}}% 
            \lccode`\~`\!\lowercase{\def~}{\discretionary{\hbox{\char`\!}}{\Wrappedafterbreak}{\hbox{\char`\!}}}% 
            \lccode`\~`\/\lowercase{\def~}{\discretionary{\hbox{\char`\/}}{\Wrappedafterbreak}{\hbox{\char`\/}}}% 
            \catcode`\.\active
            \catcode`\,\active 
            \catcode`\;\active
            \catcode`\:\active
            \catcode`\?\active
            \catcode`\!\active
            \catcode`\/\active 
            \lccode`\~`\~ 	
        }
    \makeatother

    \let\OriginalVerbatim=\Verbatim
    \makeatletter
    \renewcommand{\Verbatim}[1][1]{%
        %\parskip\z@skip
        \sbox\Wrappedcontinuationbox {\Wrappedcontinuationsymbol}%
        \sbox\Wrappedvisiblespacebox {\FV@SetupFont\Wrappedvisiblespace}%
        \def\FancyVerbFormatLine ##1{\hsize\linewidth
            \vtop{\raggedright\hyphenpenalty\z@\exhyphenpenalty\z@
                \doublehyphendemerits\z@\finalhyphendemerits\z@
                \strut ##1\strut}%
        }%
        % If the linebreak is at a space, the latter will be displayed as visible
        % space at end of first line, and a continuation symbol starts next line.
        % Stretch/shrink are however usually zero for typewriter font.
        \def\FV@Space {%
            \nobreak\hskip\z@ plus\fontdimen3\font minus\fontdimen4\font
            \discretionary{\copy\Wrappedvisiblespacebox}{\Wrappedafterbreak}
            {\kern\fontdimen2\font}%
        }%
        
        % Allow breaks at special characters using \PYG... macros.
        \Wrappedbreaksatspecials
        % Breaks at punctuation characters . , ; ? ! and / need catcode=\active 	
        \OriginalVerbatim[#1,codes*=\Wrappedbreaksatpunct]%
    }
    \makeatother

    % Exact colors from NB
    \definecolor{incolor}{HTML}{303F9F}
    \definecolor{outcolor}{HTML}{D84315}
    \definecolor{cellborder}{HTML}{CFCFCF}
    \definecolor{cellbackground}{HTML}{F7F7F7}
    
    % prompt
    \makeatletter
    \newcommand{\boxspacing}{\kern\kvtcb@left@rule\kern\kvtcb@boxsep}
    \makeatother
    \newcommand{\prompt}[4]{
        \ttfamily\llap{{\color{#2}[#3]:\hspace{3pt}#4}}\vspace{-\baselineskip}
    }
    

    
    % Prevent overflowing lines due to hard-to-break entities
    \sloppy 
    % Setup hyperref package
    \hypersetup{
      breaklinks=true,  % so long urls are correctly broken across lines
      colorlinks=true,
      urlcolor=urlcolor,
      linkcolor=linkcolor,
      citecolor=citecolor,
      }
    % Slightly bigger margins than the latex defaults
    
    \geometry{verbose,tmargin=1in,bmargin=1in,lmargin=1in,rmargin=1in}
    
    

\begin{document}
    
    \maketitle
    
    

    
    \begin{tcolorbox}[breakable, size=fbox, boxrule=1pt, pad at break*=1mm,colback=cellbackground, colframe=cellborder]
\prompt{In}{incolor}{1}{\boxspacing}
\begin{Verbatim}[commandchars=\\\{\}]
\PY{k+kn}{import} \PY{n+nn}{numpy} \PY{k}{as} \PY{n+nn}{np}
\PY{k+kn}{import} \PY{n+nn}{pandas} \PY{k}{as} \PY{n+nn}{pd}
\PY{k+kn}{import} \PY{n+nn}{matplotlib}\PY{n+nn}{.}\PY{n+nn}{pyplot} \PY{k}{as} \PY{n+nn}{plt}
\PY{k+kn}{import} \PY{n+nn}{seaborn} \PY{k}{as} \PY{n+nn}{sns}
\PY{o}{\PYZpc{}}\PY{k}{matplotlib} inline
\end{Verbatim}
\end{tcolorbox}

    \begin{tcolorbox}[breakable, size=fbox, boxrule=1pt, pad at break*=1mm,colback=cellbackground, colframe=cellborder]
\prompt{In}{incolor}{2}{\boxspacing}
\begin{Verbatim}[commandchars=\\\{\}]
\PY{k+kn}{import} \PY{n+nn}{arviz} \PY{k}{as} \PY{n+nn}{az}
\PY{k+kn}{import} \PY{n+nn}{pymc3} \PY{k}{as} \PY{n+nn}{pm}
\end{Verbatim}
\end{tcolorbox}

    \hypertarget{vasoconstriction}{%
\section{Vasoconstriction}\label{vasoconstriction}}

The data give the presence or absence (\(y_i\) = 1 or 0) of
vasoconstric- tion in the skin of the fingers following inhalation of a
certain volume of air (\(v_i\)) at a certain average rate (\(r_i\)).
Total number of records is 39. The candidate models for analyzing the
relationship are the usual logit, probit, cloglog, loglog, and cauchyit
models.

    \begin{tcolorbox}[breakable, size=fbox, boxrule=1pt, pad at break*=1mm,colback=cellbackground, colframe=cellborder]
\prompt{In}{incolor}{3}{\boxspacing}
\begin{Verbatim}[commandchars=\\\{\}]
\PY{n}{y\PYZus{}1} \PY{o}{=} \PY{n}{np}\PY{o}{.}\PY{n}{array}\PY{p}{(}\PY{p}{[}\PY{l+m+mi}{1}\PY{p}{,}\PY{l+m+mi}{1}\PY{p}{,}\PY{l+m+mi}{1}\PY{p}{,}\PY{l+m+mi}{1}\PY{p}{,}\PY{l+m+mi}{1}\PY{p}{,}\PY{l+m+mi}{1}\PY{p}{,}\PY{l+m+mi}{0}\PY{p}{,}\PY{l+m+mi}{0}\PY{p}{,}\PY{l+m+mi}{0}\PY{p}{,}\PY{l+m+mi}{0}\PY{p}{,}\PY{l+m+mi}{0}\PY{p}{,}\PY{l+m+mi}{0}\PY{p}{,}\PY{l+m+mi}{0}\PY{p}{,}\PY{l+m+mi}{1}\PY{p}{,}\PY{l+m+mi}{1}\PY{p}{,}\PY{l+m+mi}{1}\PY{p}{,}\PY{l+m+mi}{1}\PY{p}{,}\PY{l+m+mi}{1}\PY{p}{,}
   \PY{l+m+mi}{0}\PY{p}{,}\PY{l+m+mi}{1}\PY{p}{,}\PY{l+m+mi}{0}\PY{p}{,}\PY{l+m+mi}{0}\PY{p}{,}\PY{l+m+mi}{0}\PY{p}{,}\PY{l+m+mi}{0}\PY{p}{,}\PY{l+m+mi}{1}\PY{p}{,}\PY{l+m+mi}{0}\PY{p}{,}\PY{l+m+mi}{1}\PY{p}{,}\PY{l+m+mi}{0}\PY{p}{,}\PY{l+m+mi}{1}\PY{p}{,}\PY{l+m+mi}{0}\PY{p}{,}\PY{l+m+mi}{1}\PY{p}{,}\PY{l+m+mi}{0}\PY{p}{,}\PY{l+m+mi}{0}\PY{p}{,}\PY{l+m+mi}{1}\PY{p}{,}\PY{l+m+mi}{1}\PY{p}{,}\PY{l+m+mi}{1}\PY{p}{,}\PY{l+m+mi}{0}\PY{p}{,}\PY{l+m+mi}{0}\PY{p}{,}\PY{l+m+mi}{1}\PY{p}{]}\PY{p}{)}
\PY{n}{v\PYZus{}1} \PY{o}{=} \PY{n}{np}\PY{o}{.}\PY{n}{array}\PY{p}{(}\PY{p}{[}\PY{l+m+mf}{3.7}\PY{p}{,} \PY{l+m+mf}{3.5}\PY{p}{,} \PY{l+m+mf}{1.25}\PY{p}{,} \PY{l+m+mf}{0.75}\PY{p}{,} \PY{l+m+mf}{0.8}\PY{p}{,} \PY{l+m+mf}{0.7}\PY{p}{,} \PY{l+m+mf}{0.6}\PY{p}{,} \PY{l+m+mf}{1.1}\PY{p}{,} \PY{l+m+mf}{0.9}\PY{p}{,} \PY{l+m+mf}{0.9}\PY{p}{,} \PY{l+m+mf}{0.8}\PY{p}{,} \PY{l+m+mf}{0.55}\PY{p}{,} \PY{l+m+mf}{0.6}\PY{p}{,} \PY{l+m+mf}{1.4}\PY{p}{,} \PY{l+m+mf}{0.75}\PY{p}{,} \PY{l+m+mf}{2.3}\PY{p}{,} \PY{l+m+mf}{3.2}\PY{p}{,} \PY{l+m+mf}{0.85}\PY{p}{,} \PY{l+m+mf}{1.7}\PY{p}{,}
                \PY{l+m+mf}{1.8}\PY{p}{,} \PY{l+m+mf}{0.4}\PY{p}{,} \PY{l+m+mf}{0.95}\PY{p}{,} \PY{l+m+mf}{1.35}\PY{p}{,} \PY{l+m+mf}{1.5}\PY{p}{,} \PY{l+m+mf}{1.6}\PY{p}{,} \PY{l+m+mf}{0.6}\PY{p}{,} \PY{l+m+mf}{1.8}\PY{p}{,} \PY{l+m+mf}{0.95}\PY{p}{,} \PY{l+m+mf}{1.9}\PY{p}{,} \PY{l+m+mf}{1.6}\PY{p}{,} \PY{l+m+mf}{2.7}\PY{p}{,} \PY{l+m+mf}{2.35}\PY{p}{,} \PY{l+m+mf}{1.1}\PY{p}{,} \PY{l+m+mf}{1.1}\PY{p}{,} \PY{l+m+mf}{1.2}\PY{p}{,} \PY{l+m+mf}{0.8}\PY{p}{,} \PY{l+m+mf}{0.95}\PY{p}{,} \PY{l+m+mf}{0.75}\PY{p}{,} \PY{l+m+mf}{1.3}\PY{p}{]}\PY{p}{)}
\PY{n}{r\PYZus{}1} \PY{o}{=} \PY{n}{np}\PY{o}{.}\PY{n}{array}\PY{p}{(}\PY{p}{[}\PY{l+m+mf}{0.825}\PY{p}{,} \PY{l+m+mf}{1.09}\PY{p}{,} \PY{l+m+mf}{2.5}\PY{p}{,} \PY{l+m+mf}{1.5}\PY{p}{,} \PY{l+m+mf}{3.2}\PY{p}{,} \PY{l+m+mf}{3.5}\PY{p}{,} \PY{l+m+mf}{0.75}\PY{p}{,} \PY{l+m+mf}{1.7}\PY{p}{,} \PY{l+m+mf}{0.75}\PY{p}{,} \PY{l+m+mf}{0.45}\PY{p}{,} \PY{l+m+mf}{0.57}\PY{p}{,} \PY{l+m+mf}{2.75}\PY{p}{,} \PY{l+m+mi}{3}\PY{p}{,} \PY{l+m+mf}{2.33}\PY{p}{,} \PY{l+m+mf}{3.75}\PY{p}{,} \PY{l+m+mf}{1.64}\PY{p}{,} \PY{l+m+mf}{1.6}\PY{p}{,} \PY{l+m+mf}{1.415}\PY{p}{,}
\PY{l+m+mf}{1.06}\PY{p}{,} \PY{l+m+mf}{1.8}\PY{p}{,} \PY{l+m+mi}{2}\PY{p}{,} \PY{l+m+mf}{1.36}\PY{p}{,} \PY{l+m+mf}{1.35}\PY{p}{,} \PY{l+m+mf}{1.36}\PY{p}{,} \PY{l+m+mf}{1.78}\PY{p}{,} \PY{l+m+mf}{1.5}\PY{p}{,} \PY{l+m+mf}{1.5}\PY{p}{,} \PY{l+m+mf}{1.9}\PY{p}{,} \PY{l+m+mf}{0.95}\PY{p}{,} \PY{l+m+mf}{0.4}\PY{p}{,} \PY{l+m+mf}{0.75}\PY{p}{,} \PY{l+m+mf}{0.3}\PY{p}{,} \PY{l+m+mf}{1.83}\PY{p}{,} \PY{l+m+mf}{2.2}\PY{p}{,} \PY{l+m+mi}{2}\PY{p}{,} \PY{l+m+mf}{3.33}\PY{p}{,} \PY{l+m+mf}{1.9}\PY{p}{,} \PY{l+m+mf}{1.9}\PY{p}{,} \PY{l+m+mf}{1.625}\PY{p}{]}\PY{p}{)}
\end{Verbatim}
\end{tcolorbox}

    \hypertarget{a-transform-covariates-v-and-r-as-x_1-log10v-x_2-log10r.}{%
\subsection{\texorpdfstring{(a) Transform covariates \(v\) and \(r\) as
\(x_1\) = log(10×\(v\)), \(x_2\) =
log(10×\(r\)).}{(a) Transform covariates v and r as x\_1 = log(10×v), x\_2 = log(10×r).}}\label{a-transform-covariates-v-and-r-as-x_1-log10v-x_2-log10r.}}

    \begin{tcolorbox}[breakable, size=fbox, boxrule=1pt, pad at break*=1mm,colback=cellbackground, colframe=cellborder]
\prompt{In}{incolor}{4}{\boxspacing}
\begin{Verbatim}[commandchars=\\\{\}]
\PY{n}{x\PYZus{}1} \PY{o}{=} \PY{n}{np}\PY{o}{.}\PY{n}{log}\PY{p}{(}\PY{l+m+mi}{10} \PY{o}{*} \PY{n}{v\PYZus{}1}\PY{p}{)}
\PY{n}{x\PYZus{}1}
\end{Verbatim}
\end{tcolorbox}

            \begin{tcolorbox}[breakable, size=fbox, boxrule=.5pt, pad at break*=1mm, opacityfill=0]
\prompt{Out}{outcolor}{4}{\boxspacing}
\begin{Verbatim}[commandchars=\\\{\}]
array([3.61091791, 3.55534806, 2.52572864, 2.01490302, 2.07944154,
       1.94591015, 1.79175947, 2.39789527, 2.19722458, 2.19722458,
       2.07944154, 1.70474809, 1.79175947, 2.63905733, 2.01490302,
       3.13549422, 3.4657359 , 2.14006616, 2.83321334, 2.89037176,
       1.38629436, 2.2512918 , 2.60268969, 2.7080502 , 2.77258872,
       1.79175947, 2.89037176, 2.2512918 , 2.94443898, 2.77258872,
       3.29583687, 3.15700042, 2.39789527, 2.39789527, 2.48490665,
       2.07944154, 2.2512918 , 2.01490302, 2.56494936])
\end{Verbatim}
\end{tcolorbox}
        
    \begin{tcolorbox}[breakable, size=fbox, boxrule=1pt, pad at break*=1mm,colback=cellbackground, colframe=cellborder]
\prompt{In}{incolor}{5}{\boxspacing}
\begin{Verbatim}[commandchars=\\\{\}]
\PY{n}{x\PYZus{}2} \PY{o}{=} \PY{n}{np}\PY{o}{.}\PY{n}{log}\PY{p}{(}\PY{l+m+mi}{10} \PY{o}{*} \PY{n}{r\PYZus{}1}\PY{p}{)}
\PY{n}{x\PYZus{}2}
\end{Verbatim}
\end{tcolorbox}

            \begin{tcolorbox}[breakable, size=fbox, boxrule=.5pt, pad at break*=1mm, opacityfill=0]
\prompt{Out}{outcolor}{5}{\boxspacing}
\begin{Verbatim}[commandchars=\\\{\}]
array([2.1102132 , 2.38876279, 3.21887582, 2.7080502 , 3.4657359 ,
       3.55534806, 2.01490302, 2.83321334, 2.01490302, 1.5040774 ,
       1.74046617, 3.314186  , 3.40119738, 3.14845336, 3.62434093,
       2.79728133, 2.77258872, 2.64971462, 2.360854  , 2.89037176,
       2.99573227, 2.61006979, 2.60268969, 2.61006979, 2.87919846,
       2.7080502 , 2.7080502 , 2.94443898, 2.2512918 , 1.38629436,
       2.01490302, 1.09861229, 2.90690106, 3.09104245, 2.99573227,
       3.5055574 , 2.94443898, 2.94443898, 2.78809291])
\end{Verbatim}
\end{tcolorbox}
        
    \begin{tcolorbox}[breakable, size=fbox, boxrule=1pt, pad at break*=1mm,colback=cellbackground, colframe=cellborder]
\prompt{In}{incolor}{6}{\boxspacing}
\begin{Verbatim}[commandchars=\\\{\}]
\PY{n}{df} \PY{o}{=} \PY{n}{pd}\PY{o}{.}\PY{n}{DataFrame}\PY{p}{(}\PY{p}{\PYZob{}}\PY{l+s+s2}{\PYZdq{}}\PY{l+s+s2}{vasoconstriction}\PY{l+s+s2}{\PYZdq{}}\PY{p}{:} \PY{n}{y\PYZus{}1}\PY{p}{,} \PY{l+s+s2}{\PYZdq{}}\PY{l+s+s2}{air\PYZus{}log}\PY{l+s+s2}{\PYZdq{}}\PY{p}{:} \PY{n}{x\PYZus{}1}\PY{p}{,} \PY{l+s+s2}{\PYZdq{}}\PY{l+s+s2}{rate\PYZus{}log}\PY{l+s+s2}{\PYZdq{}}\PY{p}{:} \PY{n}{x\PYZus{}2}\PY{p}{\PYZcb{}}\PY{p}{)}
\PY{n}{df}\PY{o}{.}\PY{n}{head}\PY{p}{(}\PY{p}{)}
\end{Verbatim}
\end{tcolorbox}

            \begin{tcolorbox}[breakable, size=fbox, boxrule=.5pt, pad at break*=1mm, opacityfill=0]
\prompt{Out}{outcolor}{6}{\boxspacing}
\begin{Verbatim}[commandchars=\\\{\}]
   vasoconstriction   air\_log  rate\_log
0                 1  3.610918  2.110213
1                 1  3.555348  2.388763
2                 1  2.525729  3.218876
3                 1  2.014903  2.708050
4                 1  2.079442  3.465736
\end{Verbatim}
\end{tcolorbox}
        
    \hypertarget{b-estimate-posterior-means-for-coefficients-in-the-logit-model.-use-noninformative-priors-on-all-coefficients.}{%
\subsection{(b) Estimate posterior means for coefficients in the logit
model. Use noninformative priors on all
coefficients.}\label{b-estimate-posterior-means-for-coefficients-in-the-logit-model.-use-noninformative-priors-on-all-coefficients.}}

    \begin{tcolorbox}[breakable, size=fbox, boxrule=1pt, pad at break*=1mm,colback=cellbackground, colframe=cellborder]
\prompt{In}{incolor}{7}{\boxspacing}
\begin{Verbatim}[commandchars=\\\{\}]
\PY{n}{names} \PY{o}{=} \PY{n}{df}\PY{o}{.}\PY{n}{index}\PY{o}{.}\PY{n}{values}
\PY{n}{N} \PY{o}{=} \PY{n+nb}{len}\PY{p}{(}\PY{n}{names}\PY{p}{)}
\PY{n}{dims}\PY{o}{=}\PY{p}{\PYZob{}}
    \PY{l+s+s2}{\PYZdq{}}\PY{l+s+s2}{air\PYZus{}log}\PY{l+s+s2}{\PYZdq{}}\PY{p}{:} \PY{p}{[}\PY{l+s+s2}{\PYZdq{}}\PY{l+s+s2}{developer}\PY{l+s+s2}{\PYZdq{}}\PY{p}{]}\PY{p}{,}
    \PY{l+s+s2}{\PYZdq{}}\PY{l+s+s2}{rate\PYZus{}log}\PY{l+s+s2}{\PYZdq{}}\PY{p}{:} \PY{p}{[}\PY{l+s+s2}{\PYZdq{}}\PY{l+s+s2}{developer}\PY{l+s+s2}{\PYZdq{}}\PY{p}{]}
\PY{p}{\PYZcb{}}
\end{Verbatim}
\end{tcolorbox}

    \begin{tcolorbox}[breakable, size=fbox, boxrule=1pt, pad at break*=1mm,colback=cellbackground, colframe=cellborder]
\prompt{In}{incolor}{8}{\boxspacing}
\begin{Verbatim}[commandchars=\\\{\}]
\PY{l+s+sd}{\PYZsq{}\PYZsq{}\PYZsq{} }
\PY{l+s+sd}{pymc3 priors are }
\PY{l+s+sd}{default\PYZus{}regressor\PYZus{}prior = Normal.dist(mu=0, tau=1.0E\PYZhy{}6) and}
\PY{l+s+sd}{default\PYZus{}intercept\PYZus{}prior = Flat.dist()}
\PY{l+s+sd}{\PYZsq{}\PYZsq{}\PYZsq{}}
\PY{k}{with} \PY{n}{pm}\PY{o}{.}\PY{n}{Model}\PY{p}{(}\PY{p}{)} \PY{k}{as} \PY{n}{logistic\PYZus{}model}\PY{p}{:}
    \PY{n}{pm}\PY{o}{.}\PY{n}{glm}\PY{o}{.}\PY{n}{linear}\PY{o}{.}\PY{n}{GLM}\PY{p}{(}\PY{n}{y}\PY{o}{=}\PY{n}{df}\PY{p}{[}\PY{l+s+s2}{\PYZdq{}}\PY{l+s+s2}{vasoconstriction}\PY{l+s+s2}{\PYZdq{}}\PY{p}{]}\PY{p}{,} \PY{n}{x}\PY{o}{=} \PY{n}{df}\PY{p}{[}\PY{p}{[}\PY{l+s+s2}{\PYZdq{}}\PY{l+s+s2}{air\PYZus{}log}\PY{l+s+s2}{\PYZdq{}}\PY{p}{,} \PY{l+s+s2}{\PYZdq{}}\PY{l+s+s2}{rate\PYZus{}log}\PY{l+s+s2}{\PYZdq{}}\PY{p}{]}\PY{p}{]}\PY{p}{,} \PY{n}{intercept}\PY{o}{=}\PY{k+kc}{True}\PY{p}{,}
                      \PY{n}{family}\PY{o}{=}\PY{n}{pm}\PY{o}{.}\PY{n}{glm}\PY{o}{.}\PY{n}{families}\PY{o}{.}\PY{n}{Binomial}\PY{p}{(}\PY{p}{)}\PY{p}{)}
    \PY{n}{trace\PYZus{}log} \PY{o}{=} \PY{n}{pm}\PY{o}{.}\PY{n}{sample}\PY{p}{(}\PY{l+m+mi}{5000}\PY{p}{,} \PY{n}{tune}\PY{o}{=}\PY{l+m+mi}{5000}\PY{p}{,} \PY{n}{init}\PY{o}{=}\PY{l+s+s1}{\PYZsq{}}\PY{l+s+s1}{adapt\PYZus{}diag}\PY{l+s+s1}{\PYZsq{}}\PY{p}{)}
    \PY{n}{posterior\PYZus{}predictive} \PY{o}{=} \PY{n}{pm}\PY{o}{.}\PY{n}{sample\PYZus{}posterior\PYZus{}predictive}\PY{p}{(}\PY{n}{trace\PYZus{}log}\PY{p}{)}
\end{Verbatim}
\end{tcolorbox}

    \begin{Verbatim}[commandchars=\\\{\}]
Auto-assigning NUTS sampler{\ldots}
Initializing NUTS using adapt\_diag{\ldots}
Multiprocess sampling (4 chains in 4 jobs)
NUTS: [rate\_log, air\_log, Intercept]
    \end{Verbatim}

    
    \begin{verbatim}
<IPython.core.display.HTML object>
    \end{verbatim}

    
    \begin{Verbatim}[commandchars=\\\{\}]
Sampling 4 chains for 5\_000 tune and 5\_000 draw iterations (20\_000 + 20\_000
draws total) took 54 seconds.
There was 1 divergence after tuning. Increase `target\_accept` or reparameterize.
There were 26 divergences after tuning. Increase `target\_accept` or
reparameterize.
The acceptance probability does not match the target. It is 0.6210488848956406,
but should be close to 0.8. Try to increase the number of tuning steps.
There were 2 divergences after tuning. Increase `target\_accept` or
reparameterize.
The acceptance probability does not match the target. It is 0.7165668834870483,
but should be close to 0.8. Try to increase the number of tuning steps.
The number of effective samples is smaller than 10\% for some parameters.
    \end{Verbatim}

    
    \begin{verbatim}
<IPython.core.display.HTML object>
    \end{verbatim}

    
    \begin{tcolorbox}[breakable, size=fbox, boxrule=1pt, pad at break*=1mm,colback=cellbackground, colframe=cellborder]
\prompt{In}{incolor}{9}{\boxspacing}
\begin{Verbatim}[commandchars=\\\{\}]
\PY{n}{az}\PY{o}{.}\PY{n}{summary}\PY{p}{(}\PY{n}{trace\PYZus{}log}\PY{p}{)}
\end{Verbatim}
\end{tcolorbox}

    \begin{Verbatim}[commandchars=\\\{\}]
/opt/anaconda3/lib/python3.7/site-packages/arviz/data/io\_pymc3.py:91:
FutureWarning: Using `from\_pymc3` without the model will be deprecated in a
future release. Not using the model will return less accurate and less useful
results. Make sure you use the model argument or call from\_pymc3 within a model
context.
  FutureWarning,
    \end{Verbatim}

            \begin{tcolorbox}[breakable, size=fbox, boxrule=.5pt, pad at break*=1mm, opacityfill=0]
\prompt{Out}{outcolor}{9}{\boxspacing}
\begin{Verbatim}[commandchars=\\\{\}]
             mean      sd  hdi\_3\%  hdi\_97\%  mcse\_mean  mcse\_sd  ess\_mean  \textbackslash{}
Intercept -30.883  10.584 -50.062  -12.367      0.408    0.289     672.0
air\_log     6.269   2.174   2.536   10.256      0.084    0.059     671.0
rate\_log    5.595   2.021   2.366    9.520      0.073    0.051     776.0

           ess\_sd  ess\_bulk  ess\_tail  r\_hat
Intercept   672.0     519.0     322.0   1.01
air\_log     671.0     523.0     336.0   1.01
rate\_log    776.0     630.0     789.0   1.01
\end{Verbatim}
\end{tcolorbox}
        
    The means of intercept, air\_log and rate\_log are -30.883, 6.269 and
5.595, respectively.

    \hypertarget{c-for-a-subject-with-v-r-1.5-find-the-probability-of-vasoconstriction.}{%
\subsection{\texorpdfstring{(c) For a subject with \(v = r = 1.5\), find
the probability of
vasoconstriction.}{(c) For a subject with v = r = 1.5, find the probability of vasoconstriction.}}\label{c-for-a-subject-with-v-r-1.5-find-the-probability-of-vasoconstriction.}}

    \$P(vasoconstriction=1) = p \$ =
\(\frac{1}{ 1+e^(-(intercept + 6.461 \times air_log + 5.719 \times rate_log)) }\)

    \begin{tcolorbox}[breakable, size=fbox, boxrule=1pt, pad at break*=1mm,colback=cellbackground, colframe=cellborder]
\prompt{In}{incolor}{42}{\boxspacing}
\begin{Verbatim}[commandchars=\\\{\}]
\PY{n}{prob\PYZus{}v} \PY{o}{=} \PY{l+m+mi}{1}\PY{o}{/} \PY{p}{(}\PY{l+m+mi}{1} \PY{o}{+} \PY{n}{np}\PY{o}{.}\PY{n}{exp}\PY{p}{(}\PY{o}{\PYZhy{}}\PY{p}{(}\PY{o}{\PYZhy{}}\PY{l+m+mf}{30.883} \PY{o}{+} \PY{l+m+mf}{6.269} \PY{o}{*} \PY{n}{np}\PY{o}{.}\PY{n}{log}\PY{p}{(}\PY{l+m+mf}{1.5} \PY{o}{*} \PY{l+m+mi}{10}\PY{p}{)} \PY{o}{+} \PY{l+m+mf}{5.595} \PY{o}{*} \PY{n}{np}\PY{o}{.}\PY{n}{log}\PY{p}{(}\PY{l+m+mf}{1.5} \PY{o}{*} \PY{l+m+mi}{10}\PY{p}{)}\PY{p}{)}\PY{p}{)}\PY{p}{)}
\PY{n+nb}{print}\PY{p}{(}\PY{l+s+s2}{\PYZdq{}}\PY{l+s+s2}{The probability of vasoconstriction with v = r = 1.5: }\PY{l+s+s2}{\PYZdq{}}\PY{p}{,} \PY{n}{prob\PYZus{}v}\PY{p}{)}
\end{Verbatim}
\end{tcolorbox}

    \begin{Verbatim}[commandchars=\\\{\}]
The probability of vasoconstriction with v = r = 1.5:  0.7764865250020363
    \end{Verbatim}

    \hypertarget{d-compare-with-the-result-of-probit-model.-which-has-smaller-deviance}{%
\subsection{(d) Compare with the result of probit model. Which has
smaller
deviance?}\label{d-compare-with-the-result-of-probit-model.-which-has-smaller-deviance}}

    \begin{tcolorbox}[breakable, size=fbox, boxrule=1pt, pad at break*=1mm,colback=cellbackground, colframe=cellborder]
\prompt{In}{incolor}{11}{\boxspacing}
\begin{Verbatim}[commandchars=\\\{\}]
\PY{k+kn}{import} \PY{n+nn}{theano}\PY{n+nn}{.}\PY{n+nn}{tensor} \PY{k}{as} \PY{n+nn}{tsr}
\PY{k+kn}{from} \PY{n+nn}{collections} \PY{k+kn}{import} \PY{n}{OrderedDict}
\end{Verbatim}
\end{tcolorbox}

    \begin{tcolorbox}[breakable, size=fbox, boxrule=1pt, pad at break*=1mm,colback=cellbackground, colframe=cellborder]
\prompt{In}{incolor}{12}{\boxspacing}
\begin{Verbatim}[commandchars=\\\{\}]
\PY{k}{with} \PY{n}{pm}\PY{o}{.}\PY{n}{Model}\PY{p}{(}\PY{p}{)} \PY{k}{as} \PY{n}{probit\PYZus{}model}\PY{p}{:}    
    \PY{c+c1}{\PYZsh{} priors}
    \PY{n}{intercept} \PY{o}{=} \PY{n}{pm}\PY{o}{.}\PY{n}{Flat}\PY{p}{(}\PY{l+s+s2}{\PYZdq{}}\PY{l+s+s2}{intercept}\PY{l+s+s2}{\PYZdq{}}\PY{p}{)}
    \PY{n}{beta0} \PY{o}{=} \PY{n}{pm}\PY{o}{.}\PY{n}{Normal}\PY{p}{(}\PY{l+s+s2}{\PYZdq{}}\PY{l+s+s2}{beta0}\PY{l+s+s2}{\PYZdq{}}\PY{p}{,} \PY{n}{mu}\PY{o}{=}\PY{l+m+mi}{0}\PY{p}{,} \PY{n}{tau}\PY{o}{=}\PY{l+m+mf}{1.0E\PYZhy{}6}\PY{p}{)}
    \PY{n}{beta1} \PY{o}{=} \PY{n}{pm}\PY{o}{.}\PY{n}{Normal}\PY{p}{(}\PY{l+s+s1}{\PYZsq{}}\PY{l+s+s1}{beta1}\PY{l+s+s1}{\PYZsq{}}\PY{p}{,} \PY{n}{mu}\PY{o}{=}\PY{l+m+mi}{0}\PY{p}{,} \PY{n}{tau}\PY{o}{=}\PY{l+m+mf}{1.0E\PYZhy{}6}\PY{p}{)}

    \PY{c+c1}{\PYZsh{} linear predictor}
    \PY{n}{theta\PYZus{}p} \PY{o}{=} \PY{n}{intercept} \PY{o}{+} \PY{n}{beta0} \PY{o}{*} \PY{n}{df}\PY{p}{[}\PY{l+s+s2}{\PYZdq{}}\PY{l+s+s2}{air\PYZus{}log}\PY{l+s+s2}{\PYZdq{}}\PY{p}{]} \PY{o}{+} \PY{n}{beta1} \PY{o}{*} \PY{n}{df}\PY{p}{[}\PY{l+s+s2}{\PYZdq{}}\PY{l+s+s2}{rate\PYZus{}log}\PY{l+s+s2}{\PYZdq{}}\PY{p}{]}

    \PY{c+c1}{\PYZsh{} Probit transform}
    \PY{k}{def} \PY{n+nf}{probit\PYZus{}phi}\PY{p}{(}\PY{n}{x}\PY{p}{)}\PY{p}{:}
        \PY{n}{mu} \PY{o}{=} \PY{l+m+mi}{0}
        \PY{n}{sd} \PY{o}{=} \PY{l+m+mi}{1}
        \PY{k}{return} \PY{l+m+mf}{0.5} \PY{o}{*} \PY{p}{(}\PY{l+m+mi}{1} \PY{o}{+} \PY{n}{tsr}\PY{o}{.}\PY{n}{erf}\PY{p}{(}\PY{p}{(}\PY{n}{x} \PY{o}{\PYZhy{}} \PY{n}{mu}\PY{p}{)} \PY{o}{/} \PY{p}{(}\PY{n}{sd} \PY{o}{*} \PY{n}{tsr}\PY{o}{.}\PY{n}{sqrt}\PY{p}{(}\PY{l+m+mi}{2}\PY{p}{)}\PY{p}{)}\PY{p}{)}\PY{p}{)}
    
    \PY{n}{theta} \PY{o}{=} \PY{n}{probit\PYZus{}phi}\PY{p}{(}\PY{n}{theta\PYZus{}p}\PY{p}{)}


    \PY{c+c1}{\PYZsh{} likelihood}
    \PY{n}{y} \PY{o}{=} \PY{n}{pm}\PY{o}{.}\PY{n}{Bernoulli}\PY{p}{(}\PY{l+s+s1}{\PYZsq{}}\PY{l+s+s1}{y}\PY{l+s+s1}{\PYZsq{}}\PY{p}{,} \PY{n}{p}\PY{o}{=}\PY{n}{theta}\PY{p}{,} \PY{n}{observed}\PY{o}{=}\PY{n}{df}\PY{p}{[}\PY{l+s+s2}{\PYZdq{}}\PY{l+s+s2}{vasoconstriction}\PY{l+s+s2}{\PYZdq{}}\PY{p}{]}\PY{p}{)}
    
    \PY{n}{trace\PYZus{}prob} \PY{o}{=} \PY{n}{pm}\PY{o}{.}\PY{n}{sample}\PY{p}{(}\PY{l+m+mi}{5000}\PY{p}{,} \PY{n}{tune}\PY{o}{=}\PY{l+m+mi}{5000}\PY{p}{,} \PY{n}{init}\PY{o}{=}\PY{l+s+s1}{\PYZsq{}}\PY{l+s+s1}{adapt\PYZus{}diag}\PY{l+s+s1}{\PYZsq{}}\PY{p}{)}
    \PY{n}{posterior\PYZus{}predictive\PYZus{}prob} \PY{o}{=} \PY{n}{pm}\PY{o}{.}\PY{n}{sample\PYZus{}posterior\PYZus{}predictive}\PY{p}{(}\PY{n}{trace\PYZus{}prob}\PY{p}{)}
\end{Verbatim}
\end{tcolorbox}

    \begin{Verbatim}[commandchars=\\\{\}]
Auto-assigning NUTS sampler{\ldots}
Initializing NUTS using adapt\_diag{\ldots}
ERROR (theano.gof.opt): Optimization failure due to: local\_grad\_log\_erfc\_neg
ERROR (theano.gof.opt): node:
Elemwise\{true\_div,no\_inplace\}(Elemwise\{mul,no\_inplace\}.0,
Elemwise\{erfc,no\_inplace\}.0)
ERROR (theano.gof.opt): TRACEBACK:
ERROR (theano.gof.opt): Traceback (most recent call last):
  File "/opt/anaconda3/lib/python3.7/site-packages/theano/gof/opt.py", line
2034, in process\_node
    replacements = lopt.transform(node)
  File "/opt/anaconda3/lib/python3.7/site-packages/theano/tensor/opt.py", line
6789, in local\_grad\_log\_erfc\_neg
    if not exp.owner.inputs[0].owner:
AttributeError: 'NoneType' object has no attribute 'owner'

Multiprocess sampling (4 chains in 4 jobs)
NUTS: [beta1, beta0, intercept]
    \end{Verbatim}

    
    \begin{verbatim}
<IPython.core.display.HTML object>
    \end{verbatim}

    
    \begin{Verbatim}[commandchars=\\\{\}]
Sampling 4 chains for 5\_000 tune and 5\_000 draw iterations (20\_000 + 20\_000
draws total) took 55 seconds.
There were 518 divergences after tuning. Increase `target\_accept` or
reparameterize.
The acceptance probability does not match the target. It is 0.6682940921512561,
but should be close to 0.8. Try to increase the number of tuning steps.
The acceptance probability does not match the target. It is 0.906901957639155,
but should be close to 0.8. Try to increase the number of tuning steps.
The number of effective samples is smaller than 10\% for some parameters.
    \end{Verbatim}

    
    \begin{verbatim}
<IPython.core.display.HTML object>
    \end{verbatim}

    
    \begin{tcolorbox}[breakable, size=fbox, boxrule=1pt, pad at break*=1mm,colback=cellbackground, colframe=cellborder]
\prompt{In}{incolor}{13}{\boxspacing}
\begin{Verbatim}[commandchars=\\\{\}]
\PY{n}{model\PYZus{}trace\PYZus{}dict} \PY{o}{=} \PY{p}{\PYZob{}}\PY{l+s+s2}{\PYZdq{}}\PY{l+s+s2}{logistic\PYZus{}model}\PY{l+s+s2}{\PYZdq{}}\PY{p}{:}\PY{n}{trace\PYZus{}log}\PY{p}{,} \PY{l+s+s2}{\PYZdq{}}\PY{l+s+s2}{probit\PYZus{}model}\PY{l+s+s2}{\PYZdq{}}\PY{p}{:} \PY{n}{trace\PYZus{}prob}\PY{p}{\PYZcb{}}
\PY{n}{az}\PY{o}{.}\PY{n}{compare}\PY{p}{(}\PY{n}{model\PYZus{}trace\PYZus{}dict}\PY{p}{,} \PY{n}{ic}\PY{o}{=}\PY{l+s+s1}{\PYZsq{}}\PY{l+s+s1}{WAIC}\PY{l+s+s1}{\PYZsq{}}\PY{p}{,} \PY{n}{scale}\PY{o}{=}\PY{l+s+s2}{\PYZdq{}}\PY{l+s+s2}{deviance}\PY{l+s+s2}{\PYZdq{}}\PY{p}{)}
\end{Verbatim}
\end{tcolorbox}

    \begin{Verbatim}[commandchars=\\\{\}]
/opt/anaconda3/lib/python3.7/site-packages/arviz/data/io\_pymc3.py:91:
FutureWarning: Using `from\_pymc3` without the model will be deprecated in a
future release. Not using the model will return less accurate and less useful
results. Make sure you use the model argument or call from\_pymc3 within a model
context.
  FutureWarning,
/opt/anaconda3/lib/python3.7/site-packages/arviz/stats/stats.py:1427:
UserWarning: For one or more samples the posterior variance of the log
predictive densities exceeds 0.4. This could be indication of WAIC starting to
fail.
See http://arxiv.org/abs/1507.04544 for details
  "For one or more samples the posterior variance of the log predictive "
/opt/anaconda3/lib/python3.7/site-packages/arviz/data/io\_pymc3.py:91:
FutureWarning: Using `from\_pymc3` without the model will be deprecated in a
future release. Not using the model will return less accurate and less useful
results. Make sure you use the model argument or call from\_pymc3 within a model
context.
  FutureWarning,
/opt/anaconda3/lib/python3.7/site-packages/arviz/stats/stats.py:1427:
UserWarning: For one or more samples the posterior variance of the log
predictive densities exceeds 0.4. This could be indication of WAIC starting to
fail.
See http://arxiv.org/abs/1507.04544 for details
  "For one or more samples the posterior variance of the log predictive "
    \end{Verbatim}

            \begin{tcolorbox}[breakable, size=fbox, boxrule=.5pt, pad at break*=1mm, opacityfill=0]
\prompt{Out}{outcolor}{13}{\boxspacing}
\begin{Verbatim}[commandchars=\\\{\}]
               rank     waic   p\_waic   d\_waic    weight       se      dse  \textbackslash{}
probit\_model      0   36.107  3.18883        0  0.601672  9.70566        0
logistic\_model    1  37.1122   3.9323  1.00522  0.398328  8.57837  1.59664

               warning waic\_scale
probit\_model      True   deviance
logistic\_model    True   deviance
\end{Verbatim}
\end{tcolorbox}
        
    Based on the above table, we could see the probit model has smaller
deviance than the logistic model.

    \hypertarget{magnesium-ammonium-phosphate-and-chrysanthemums}{%
\section{Magnesium Ammonium Phosphate and
Chrysanthemums}\label{magnesium-ammonium-phosphate-and-chrysanthemums}}

Walpole et al.~(2007) provide data from a study on the effect of
magnesium ammonium phosphate on the height of chrysanthemums, which was
conducted at George Mason University in order to determine a possible
optimum level of fertilization, based on the enhanced vertical growth
response of the chrysanthemums. Forty chrysanthemum seedlings were
assigned to 4 groups, each containing 10 plants. Each was planted in a
similar pot containing a uniform growth medium. An increasing
concentration of MgNH4PO4, measured in grams per bushel, was added to
each plant. The 4 groups of plants were grown under uniform conditions
in a greenhouse for a period of 4 weeks.

The treatments and the respective changes in heights, measured in
centimeters, are given in the following table:

    \begin{tcolorbox}[breakable, size=fbox, boxrule=1pt, pad at break*=1mm,colback=cellbackground, colframe=cellborder]
\prompt{In}{incolor}{14}{\boxspacing}
\begin{Verbatim}[commandchars=\\\{\}]
\PY{n}{g50} \PY{o}{=} \PY{n}{np}\PY{o}{.}\PY{n}{array}\PY{p}{(}\PY{p}{[}\PY{l+m+mf}{13.2}\PY{p}{,} \PY{l+m+mf}{12.4}\PY{p}{,} \PY{l+m+mf}{12.8}\PY{p}{,} \PY{l+m+mf}{17.2}\PY{p}{,} \PY{l+m+mf}{13.0}\PY{p}{,} \PY{l+m+mf}{14.0}\PY{p}{,} \PY{l+m+mf}{14.2}\PY{p}{,} \PY{l+m+mf}{21.6}\PY{p}{,} \PY{l+m+mf}{15.0}\PY{p}{,} \PY{l+m+mf}{20.0}\PY{p}{]}\PY{p}{)}
\PY{n}{g100} \PY{o}{=} \PY{n}{np}\PY{o}{.}\PY{n}{array}\PY{p}{(}\PY{p}{[}\PY{l+m+mf}{16.0}\PY{p}{,} \PY{l+m+mf}{12.6}\PY{p}{,} \PY{l+m+mf}{14.8}\PY{p}{,} \PY{l+m+mf}{13.0}\PY{p}{,} \PY{l+m+mf}{14.0}\PY{p}{,} \PY{l+m+mf}{23.6}\PY{p}{,} \PY{l+m+mf}{14.0}\PY{p}{,} \PY{l+m+mf}{17.0}\PY{p}{,} \PY{l+m+mf}{22.2}\PY{p}{,} \PY{l+m+mf}{24.4}\PY{p}{]}\PY{p}{)}
\PY{n}{g200} \PY{o}{=} \PY{n}{np}\PY{o}{.}\PY{n}{array}\PY{p}{(}\PY{p}{[}\PY{l+m+mf}{7.8}\PY{p}{,} \PY{l+m+mf}{14.4}\PY{p}{,} \PY{l+m+mf}{20.0}\PY{p}{,} \PY{l+m+mf}{15.8}\PY{p}{,} \PY{l+m+mf}{17.0}\PY{p}{,} \PY{l+m+mf}{27.0}\PY{p}{,} \PY{l+m+mf}{19.6}\PY{p}{,} \PY{l+m+mf}{18.0}\PY{p}{,} \PY{l+m+mf}{20.2}\PY{p}{,} \PY{l+m+mf}{23.2}\PY{p}{]}\PY{p}{)}
\PY{n}{g400} \PY{o}{=} \PY{n}{np}\PY{o}{.}\PY{n}{array}\PY{p}{(}\PY{p}{[}\PY{l+m+mf}{21.0}\PY{p}{,} \PY{l+m+mf}{14.8}\PY{p}{,} \PY{l+m+mf}{19.1}\PY{p}{,} \PY{l+m+mf}{15.8}\PY{p}{,} \PY{l+m+mf}{18.0}\PY{p}{,} \PY{l+m+mf}{26.0}\PY{p}{,} \PY{l+m+mf}{21.1}\PY{p}{,} \PY{l+m+mf}{22.0}\PY{p}{,} \PY{l+m+mf}{25.0}\PY{p}{,} \PY{l+m+mf}{18.2}\PY{p}{]}\PY{p}{)}
\end{Verbatim}
\end{tcolorbox}

    Solve the problem as a Bayesian one-way ANOVA. Use STZ constraints on
treatment effects.

    \hypertarget{a-do-different-concentrations-of-mgnh4po4-affect-the-average-attained-height-of-chrysanthemums-look-at-the-95-credible-sets-for-the-differences-between-treatment-effects.}{%
\subsection{(a) Do different concentrations of MgNH4PO4 affect the
average attained height of chrysanthemums? Look at the 95\% credible
sets for the differences between treatment
effects.}\label{a-do-different-concentrations-of-mgnh4po4-affect-the-average-attained-height-of-chrysanthemums-look-at-the-95-credible-sets-for-the-differences-between-treatment-effects.}}

    Explanation on choices of priors:

In pymc3, if the priors are too non-informative, the sampling will fail
because there are many zeros when the program takes derivatives.
Therefore, we released the constraints and chose weakly informative
priors on the mean, alpha2, alpha3 and alpha4.

For sigma, the best scenario is using Inversegamma with alpha = 0.0001
and beta = 0.0001. However, the same thing happened. So we released the
constraints, too. We changed alpha and beta to 0.01 and 0.01,
respectively.

    \begin{tcolorbox}[breakable, size=fbox, boxrule=1pt, pad at break*=1mm,colback=cellbackground, colframe=cellborder]
\prompt{In}{incolor}{15}{\boxspacing}
\begin{Verbatim}[commandchars=\\\{\}]
\PY{k}{with} \PY{n}{pm}\PY{o}{.}\PY{n}{Model}\PY{p}{(}\PY{p}{)} \PY{k}{as} \PY{n}{ANOVA}\PY{p}{:}
    \PY{n}{mu} \PY{o}{=} \PY{n}{pm}\PY{o}{.}\PY{n}{Normal}\PY{p}{(}\PY{l+s+s2}{\PYZdq{}}\PY{l+s+s2}{mu}\PY{l+s+s2}{\PYZdq{}}\PY{p}{,} \PY{n}{mu}\PY{o}{=}\PY{l+m+mi}{0}\PY{p}{,} \PY{n}{sigma}\PY{o}{=}\PY{l+m+mi}{1}\PY{p}{)}
    \PY{n}{sigma} \PY{o}{=} \PY{n}{pm}\PY{o}{.}\PY{n}{InverseGamma}\PY{p}{(}\PY{l+s+s2}{\PYZdq{}}\PY{l+s+s2}{sigma}\PY{l+s+s2}{\PYZdq{}}\PY{p}{,} \PY{n}{alpha}\PY{o}{=}\PY{l+m+mf}{0.01}\PY{p}{,} \PY{n}{beta}\PY{o}{=}\PY{l+m+mf}{0.01}\PY{p}{)}
    \PY{n}{alpha2} \PY{o}{=} \PY{n}{pm}\PY{o}{.}\PY{n}{Normal}\PY{p}{(}\PY{l+s+s2}{\PYZdq{}}\PY{l+s+s2}{alpha2}\PY{l+s+s2}{\PYZdq{}}\PY{p}{,} \PY{n}{mu}\PY{o}{=}\PY{l+m+mi}{0}\PY{p}{,} \PY{n}{sigma}\PY{o}{=}\PY{l+m+mi}{1}\PY{p}{)}
    \PY{n}{alpha3} \PY{o}{=} \PY{n}{pm}\PY{o}{.}\PY{n}{Normal}\PY{p}{(}\PY{l+s+s2}{\PYZdq{}}\PY{l+s+s2}{alpha3}\PY{l+s+s2}{\PYZdq{}}\PY{p}{,} \PY{n}{mu}\PY{o}{=}\PY{l+m+mi}{0}\PY{p}{,} \PY{n}{sigma}\PY{o}{=}\PY{l+m+mi}{1}\PY{p}{)}
    \PY{n}{alpha4} \PY{o}{=} \PY{n}{pm}\PY{o}{.}\PY{n}{Normal}\PY{p}{(}\PY{l+s+s2}{\PYZdq{}}\PY{l+s+s2}{alpha4}\PY{l+s+s2}{\PYZdq{}}\PY{p}{,} \PY{n}{mu}\PY{o}{=}\PY{l+m+mi}{0}\PY{p}{,} \PY{n}{sigma}\PY{o}{=}\PY{l+m+mi}{1}\PY{p}{)}
    \PY{c+c1}{\PYZsh{} STZ constraints}
    \PY{n}{alpha1} \PY{o}{=} \PY{n}{pm}\PY{o}{.}\PY{n}{Deterministic}\PY{p}{(}\PY{l+s+s2}{\PYZdq{}}\PY{l+s+s2}{alpha1}\PY{l+s+s2}{\PYZdq{}}\PY{p}{,} \PY{o}{\PYZhy{}}\PY{p}{(}\PY{n}{alpha2}\PY{o}{+}\PY{n}{alpha3}\PY{o}{+}\PY{n}{alpha4}\PY{p}{)}\PY{p}{)}
    
    \PY{c+c1}{\PYZsh{} likelihood}
    \PY{n}{treatment\PYZus{}50} \PY{o}{=} \PY{n}{pm}\PY{o}{.}\PY{n}{Normal}\PY{p}{(}\PY{l+s+s2}{\PYZdq{}}\PY{l+s+s2}{g50}\PY{l+s+s2}{\PYZdq{}}\PY{p}{,} \PY{n}{mu}\PY{o}{=}\PY{n}{mu}\PY{o}{+}\PY{n}{alpha1}\PY{p}{,} \PY{n}{sigma}\PY{o}{=}\PY{n}{sigma}\PY{p}{,} \PY{n}{observed}\PY{o}{=}\PY{n}{g50}\PY{p}{)}
    \PY{n}{treatment\PYZus{}100} \PY{o}{=} \PY{n}{pm}\PY{o}{.}\PY{n}{Normal}\PY{p}{(}\PY{l+s+s2}{\PYZdq{}}\PY{l+s+s2}{g100}\PY{l+s+s2}{\PYZdq{}}\PY{p}{,} \PY{n}{mu}\PY{o}{=}\PY{n}{mu}\PY{o}{+}\PY{n}{alpha2}\PY{p}{,} \PY{n}{sigma}\PY{o}{=}\PY{n}{sigma}\PY{p}{,} \PY{n}{observed}\PY{o}{=}\PY{n}{g100}\PY{p}{)}
    \PY{n}{treatment\PYZus{}200} \PY{o}{=} \PY{n}{pm}\PY{o}{.}\PY{n}{Normal}\PY{p}{(}\PY{l+s+s2}{\PYZdq{}}\PY{l+s+s2}{g200}\PY{l+s+s2}{\PYZdq{}}\PY{p}{,} \PY{n}{mu}\PY{o}{=}\PY{n}{mu}\PY{o}{+}\PY{n}{alpha3}\PY{p}{,} \PY{n}{sigma}\PY{o}{=}\PY{n}{sigma}\PY{p}{,} \PY{n}{observed}\PY{o}{=}\PY{n}{g200}\PY{p}{)}
    \PY{n}{treatment\PYZus{}400} \PY{o}{=} \PY{n}{pm}\PY{o}{.}\PY{n}{Normal}\PY{p}{(}\PY{l+s+s2}{\PYZdq{}}\PY{l+s+s2}{g400}\PY{l+s+s2}{\PYZdq{}}\PY{p}{,} \PY{n}{mu}\PY{o}{=}\PY{n}{mu}\PY{o}{+}\PY{n}{alpha4}\PY{p}{,} \PY{n}{sigma}\PY{o}{=}\PY{n}{sigma}\PY{p}{,} \PY{n}{observed}\PY{o}{=}\PY{n}{g400}\PY{p}{)}
\end{Verbatim}
\end{tcolorbox}

    \begin{tcolorbox}[breakable, size=fbox, boxrule=1pt, pad at break*=1mm,colback=cellbackground, colframe=cellborder]
\prompt{In}{incolor}{16}{\boxspacing}
\begin{Verbatim}[commandchars=\\\{\}]
\PY{k}{with} \PY{n}{ANOVA}\PY{p}{:}
    \PY{n}{alpha1\PYZus{}diff\PYZus{}alpha2} \PY{o}{=} \PY{n}{pm}\PY{o}{.}\PY{n}{Deterministic}\PY{p}{(}\PY{l+s+s1}{\PYZsq{}}\PY{l+s+s1}{a1\PYZhy{}a2}\PY{l+s+s1}{\PYZsq{}}\PY{p}{,} \PY{n}{alpha1} \PY{o}{\PYZhy{}} \PY{n}{alpha2}\PY{p}{)}
    \PY{n}{alpha1\PYZus{}diff\PYZus{}alpha3} \PY{o}{=} \PY{n}{pm}\PY{o}{.}\PY{n}{Deterministic}\PY{p}{(}\PY{l+s+s1}{\PYZsq{}}\PY{l+s+s1}{a1\PYZhy{}a3}\PY{l+s+s1}{\PYZsq{}}\PY{p}{,} \PY{n}{alpha1} \PY{o}{\PYZhy{}} \PY{n}{alpha3}\PY{p}{)}
    \PY{n}{alpha1\PYZus{}diff\PYZus{}alpha4} \PY{o}{=} \PY{n}{pm}\PY{o}{.}\PY{n}{Deterministic}\PY{p}{(}\PY{l+s+s1}{\PYZsq{}}\PY{l+s+s1}{a1\PYZhy{}a4}\PY{l+s+s1}{\PYZsq{}}\PY{p}{,} \PY{n}{alpha1} \PY{o}{\PYZhy{}} \PY{n}{alpha4}\PY{p}{)}
    
    \PY{n}{alpha2\PYZus{}diff\PYZus{}alpha3} \PY{o}{=} \PY{n}{pm}\PY{o}{.}\PY{n}{Deterministic}\PY{p}{(}\PY{l+s+s1}{\PYZsq{}}\PY{l+s+s1}{a2\PYZhy{}a3}\PY{l+s+s1}{\PYZsq{}}\PY{p}{,} \PY{n}{alpha2} \PY{o}{\PYZhy{}} \PY{n}{alpha3}\PY{p}{)}
    \PY{n}{alpha2\PYZus{}diff\PYZus{}alpha4} \PY{o}{=} \PY{n}{pm}\PY{o}{.}\PY{n}{Deterministic}\PY{p}{(}\PY{l+s+s1}{\PYZsq{}}\PY{l+s+s1}{a2\PYZhy{}a4}\PY{l+s+s1}{\PYZsq{}}\PY{p}{,} \PY{n}{alpha2} \PY{o}{\PYZhy{}} \PY{n}{alpha4}\PY{p}{)}
    \PY{n}{alpha3\PYZus{}diff\PYZus{}alpha4} \PY{o}{=} \PY{n}{pm}\PY{o}{.}\PY{n}{Deterministic}\PY{p}{(}\PY{l+s+s1}{\PYZsq{}}\PY{l+s+s1}{a3\PYZhy{}a4}\PY{l+s+s1}{\PYZsq{}}\PY{p}{,} \PY{n}{alpha3} \PY{o}{\PYZhy{}} \PY{n}{alpha4}\PY{p}{)}    
\end{Verbatim}
\end{tcolorbox}

    \begin{tcolorbox}[breakable, size=fbox, boxrule=1pt, pad at break*=1mm,colback=cellbackground, colframe=cellborder]
\prompt{In}{incolor}{17}{\boxspacing}
\begin{Verbatim}[commandchars=\\\{\}]
\PY{k}{with} \PY{n}{ANOVA}\PY{p}{:}
    \PY{n}{trace\PYZus{}anova} \PY{o}{=} \PY{n}{pm}\PY{o}{.}\PY{n}{sample}\PY{p}{(}\PY{l+m+mi}{5000}\PY{p}{,} \PY{n}{tune}\PY{o}{=}\PY{l+m+mi}{5000}\PY{p}{,} \PY{n}{init}\PY{o}{=}\PY{l+s+s1}{\PYZsq{}}\PY{l+s+s1}{adapt\PYZus{}diag}\PY{l+s+s1}{\PYZsq{}}\PY{p}{)}
\end{Verbatim}
\end{tcolorbox}

    \begin{Verbatim}[commandchars=\\\{\}]
Auto-assigning NUTS sampler{\ldots}
Initializing NUTS using adapt\_diag{\ldots}
Multiprocess sampling (4 chains in 4 jobs)
NUTS: [alpha4, alpha3, alpha2, sigma, mu]
    \end{Verbatim}

    
    \begin{verbatim}
<IPython.core.display.HTML object>
    \end{verbatim}

    
    \begin{Verbatim}[commandchars=\\\{\}]
Sampling 4 chains for 5\_000 tune and 5\_000 draw iterations (20\_000 + 20\_000
draws total) took 30 seconds.
    \end{Verbatim}

    \begin{tcolorbox}[breakable, size=fbox, boxrule=1pt, pad at break*=1mm,colback=cellbackground, colframe=cellborder]
\prompt{In}{incolor}{18}{\boxspacing}
\begin{Verbatim}[commandchars=\\\{\}]
\PY{n}{az}\PY{o}{.}\PY{n}{summary}\PY{p}{(}\PY{n}{trace\PYZus{}anova}\PY{p}{,} \PY{n}{stat\PYZus{}funcs}\PY{o}{=}\PY{p}{\PYZob{}}\PY{l+s+s2}{\PYZdq{}}\PY{l+s+s2}{hdi\PYZus{}2.5}\PY{l+s+s2}{\PYZpc{}}\PY{l+s+s2}{\PYZdq{}}\PY{p}{:} \PY{k}{lambda} \PY{n}{x}\PY{p}{:}\PY{n}{np}\PY{o}{.}\PY{n}{percentile}\PY{p}{(}\PY{n}{x}\PY{p}{,} \PY{l+m+mf}{2.5}\PY{p}{)}\PY{p}{,} \PY{l+s+s2}{\PYZdq{}}\PY{l+s+s2}{hdi\PYZus{}97.5}\PY{l+s+s2}{\PYZpc{}}\PY{l+s+s2}{\PYZdq{}}\PY{p}{:} \PY{k}{lambda} \PY{n}{x} \PY{p}{:} \PY{n}{np}\PY{o}{.}\PY{n}{percentile}\PY{p}{(}\PY{n}{x}\PY{p}{,} \PY{l+m+mf}{97.5}\PY{p}{)}\PY{p}{\PYZcb{}}\PY{p}{)}
\end{Verbatim}
\end{tcolorbox}

    \begin{Verbatim}[commandchars=\\\{\}]
/opt/anaconda3/lib/python3.7/site-packages/arviz/data/io\_pymc3.py:91:
FutureWarning: Using `from\_pymc3` without the model will be deprecated in a
future release. Not using the model will return less accurate and less useful
results. Make sure you use the model argument or call from\_pymc3 within a model
context.
  FutureWarning,
    \end{Verbatim}

            \begin{tcolorbox}[breakable, size=fbox, boxrule=.5pt, pad at break*=1mm, opacityfill=0]
\prompt{Out}{outcolor}{18}{\boxspacing}
\begin{Verbatim}[commandchars=\\\{\}]
          mean     sd  hdi\_3\%  hdi\_97\%  mcse\_mean  mcse\_sd  ess\_mean   ess\_sd  \textbackslash{}
mu       2.427  1.085   0.378    4.434      0.008    0.006   17454.0  16639.0
alpha2   0.057  0.972  -1.689    1.970      0.006    0.007   27958.0  10118.0
alpha3   0.102  0.962  -1.654    1.953      0.006    0.007   27518.0  10682.0
alpha4   0.158  0.964  -1.661    1.958      0.006    0.007   30313.0  10545.0
sigma   16.241  2.138  12.448   20.348      0.016    0.012   17266.0  17266.0
alpha1  -0.317  1.601  -3.270    2.760      0.009    0.011   30737.0  11046.0
a1-a2   -0.374  2.285  -4.663    3.914      0.013    0.016   30231.0  10665.0
a1-a3   -0.419  2.274  -4.599    3.991      0.013    0.015   29452.0  11092.0
a1-a4   -0.475  2.287  -4.822    3.769      0.013    0.016   31153.0  10730.0
a2-a3   -0.045  1.409  -2.618    2.672      0.009    0.010   26664.0  10495.0
a2-a4   -0.101  1.393  -2.652    2.561      0.009    0.010   26810.0  10476.0
a3-a4   -0.055  1.384  -2.607    2.570      0.008    0.010   28759.0  10374.0

        ess\_bulk  ess\_tail  r\_hat  hdi\_2.5\%  hdi\_97.5\%
mu       17461.0   14601.0    1.0     0.321      4.553
alpha2   27981.0   15530.0    1.0    -1.856      1.959
alpha3   27485.0   15232.0    1.0    -1.766      1.996
alpha4   30314.0   15572.0    1.0    -1.760      2.033
sigma    17022.0   14461.0    1.0    12.538     20.935
alpha1   30758.0   15896.0    1.0    -3.440      2.841
a1-a2    30239.0   15848.0    1.0    -4.849      4.110
a1-a3    29465.0   15846.0    1.0    -4.925      4.059
a1-a4    31199.0   15740.0    1.0    -4.911      4.056
a2-a3    26658.0   15470.0    1.0    -2.813      2.714
a2-a4    26801.0   15545.0    1.0    -2.809      2.621
a3-a4    28798.0   15198.0    1.0    -2.734      2.654
\end{Verbatim}
\end{tcolorbox}
        
    Do different concentrations of MgNH4PO4 affect the average attained
height of chrysanthemums?

Based on the above table, we could see the differences between alphas.
These differences all cover 0 which means different treatments more
likely do not affect the average attained height of chrysanthemums.

    \hypertarget{b-find-the-95-credible-set-for-the-contrast-ux3bc1-ux3bc2-ux3bc3-ux3bc4.}{%
\subsection{(b) Find the 95\% credible set for the contrast μ1 − μ2 − μ3
+
μ4.}\label{b-find-the-95-credible-set-for-the-contrast-ux3bc1-ux3bc2-ux3bc3-ux3bc4.}}

    \begin{tcolorbox}[breakable, size=fbox, boxrule=1pt, pad at break*=1mm,colback=cellbackground, colframe=cellborder]
\prompt{In}{incolor}{19}{\boxspacing}
\begin{Verbatim}[commandchars=\\\{\}]
\PY{k}{with} \PY{n}{ANOVA}\PY{p}{:}
    \PY{n}{mu\PYZus{}differences} \PY{o}{=} \PY{n}{pm}\PY{o}{.}\PY{n}{Deterministic}\PY{p}{(}\PY{l+s+s2}{\PYZdq{}}\PY{l+s+s2}{μ1−μ2−μ3+μ4}\PY{l+s+s2}{\PYZdq{}}\PY{p}{,} \PY{p}{(}\PY{n}{mu}\PY{o}{+}\PY{n}{alpha1}\PY{p}{)} \PY{o}{\PYZhy{}} \PY{p}{(}\PY{n}{mu}\PY{o}{+}\PY{n}{alpha2}\PY{p}{)} \PY{o}{\PYZhy{}} \PY{p}{(}\PY{n}{mu}\PY{o}{+}\PY{n}{alpha3}\PY{p}{)} \PY{o}{+} \PY{p}{(}\PY{n}{mu}\PY{o}{+}\PY{n}{alpha4}\PY{p}{)} \PY{p}{)}
    \PY{n}{trace\PYZus{}mu\PYZus{}differences} \PY{o}{=} \PY{n}{pm}\PY{o}{.}\PY{n}{sample}\PY{p}{(}\PY{l+m+mi}{5000}\PY{p}{,} \PY{n}{tune}\PY{o}{=}\PY{l+m+mi}{5000}\PY{p}{,} \PY{n}{init}\PY{o}{=}\PY{l+s+s1}{\PYZsq{}}\PY{l+s+s1}{adapt\PYZus{}diag}\PY{l+s+s1}{\PYZsq{}}\PY{p}{)}
\end{Verbatim}
\end{tcolorbox}

    \begin{Verbatim}[commandchars=\\\{\}]
Auto-assigning NUTS sampler{\ldots}
Initializing NUTS using adapt\_diag{\ldots}
Multiprocess sampling (4 chains in 4 jobs)
NUTS: [alpha4, alpha3, alpha2, sigma, mu]
    \end{Verbatim}

    
    \begin{verbatim}
<IPython.core.display.HTML object>
    \end{verbatim}

    
    \begin{Verbatim}[commandchars=\\\{\}]
Sampling 4 chains for 5\_000 tune and 5\_000 draw iterations (20\_000 + 20\_000
draws total) took 32 seconds.
    \end{Verbatim}

    \begin{tcolorbox}[breakable, size=fbox, boxrule=1pt, pad at break*=1mm,colback=cellbackground, colframe=cellborder]
\prompt{In}{incolor}{20}{\boxspacing}
\begin{Verbatim}[commandchars=\\\{\}]
\PY{n}{az}\PY{o}{.}\PY{n}{summary}\PY{p}{(}\PY{n}{trace\PYZus{}mu\PYZus{}differences}\PY{p}{,} \PY{n}{stat\PYZus{}funcs}\PY{o}{=}\PY{p}{\PYZob{}}\PY{l+s+s2}{\PYZdq{}}\PY{l+s+s2}{hdi\PYZus{}2.5}\PY{l+s+s2}{\PYZpc{}}\PY{l+s+s2}{\PYZdq{}}\PY{p}{:} \PY{k}{lambda} \PY{n}{x}\PY{p}{:}\PY{n}{np}\PY{o}{.}\PY{n}{percentile}\PY{p}{(}\PY{n}{x}\PY{p}{,} \PY{l+m+mf}{2.5}\PY{p}{)}\PY{p}{,} \PY{l+s+s2}{\PYZdq{}}\PY{l+s+s2}{hdi\PYZus{}97.5}\PY{l+s+s2}{\PYZpc{}}\PY{l+s+s2}{\PYZdq{}}\PY{p}{:} \PY{k}{lambda} \PY{n}{x} \PY{p}{:} \PY{n}{np}\PY{o}{.}\PY{n}{percentile}\PY{p}{(}\PY{n}{x}\PY{p}{,} \PY{l+m+mf}{97.5}\PY{p}{)}\PY{p}{\PYZcb{}}\PY{p}{)}
\end{Verbatim}
\end{tcolorbox}

    \begin{Verbatim}[commandchars=\\\{\}]
/opt/anaconda3/lib/python3.7/site-packages/arviz/data/io\_pymc3.py:91:
FutureWarning: Using `from\_pymc3` without the model will be deprecated in a
future release. Not using the model will return less accurate and less useful
results. Make sure you use the model argument or call from\_pymc3 within a model
context.
  FutureWarning,
    \end{Verbatim}

            \begin{tcolorbox}[breakable, size=fbox, boxrule=.5pt, pad at break*=1mm, opacityfill=0]
\prompt{Out}{outcolor}{20}{\boxspacing}
\begin{Verbatim}[commandchars=\\\{\}]
               mean     sd  hdi\_3\%  hdi\_97\%  mcse\_mean  mcse\_sd  ess\_mean  \textbackslash{}
mu            2.431  1.070   0.479    4.483      0.008    0.006   18756.0
alpha2        0.052  0.971  -1.748    1.897      0.006    0.007   27182.0
alpha3        0.104  0.978  -1.703    1.947      0.006    0.007   26466.0
alpha4        0.162  0.971  -1.652    1.978      0.007    0.006   20756.0
sigma        16.228  2.131  12.434   20.276      0.016    0.011   17872.0
alpha1       -0.318  1.634  -3.391    2.697      0.010    0.011   26951.0
a1-a2        -0.370  2.324  -4.827    3.944      0.014    0.016   27645.0
a1-a3        -0.423  2.329  -4.769    3.881      0.014    0.016   27860.0
a1-a4        -0.481  2.315  -4.933    3.740      0.015    0.015   24277.0
a2-a3        -0.052  1.392  -2.641    2.575      0.009    0.010   26296.0
a2-a4        -0.110  1.395  -2.741    2.486      0.009    0.010   22736.0
a3-a4        -0.058  1.404  -2.758    2.511      0.009    0.010   21936.0
μ1−μ2−μ3+μ4  -0.312  2.730  -5.334    4.952      0.017    0.019   27359.0

              ess\_sd  ess\_bulk  ess\_tail  r\_hat  hdi\_2.5\%  hdi\_97.5\%
mu           17889.0   18746.0   16089.0    1.0     0.331      4.497
alpha2        9812.0   27181.0   15304.0    1.0    -1.862      1.946
alpha3       10612.0   26470.0   14907.0    1.0    -1.800      2.007
alpha4       11464.0   20765.0   15920.0    1.0    -1.721      2.068
sigma        17872.0   17655.0   14169.0    1.0    12.573     20.920
alpha1       10936.0   26919.0   15512.0    1.0    -3.510      2.845
a1-a2        10479.0   27633.0   15421.0    1.0    -4.955      4.153
a1-a3        10827.0   27869.0   15191.0    1.0    -4.938      4.088
a1-a4        11300.0   24279.0   15715.0    1.0    -5.008      4.065
a2-a3        10330.0   26357.0   15717.0    1.0    -2.786      2.668
a2-a4        10619.0   22738.0   15147.0    1.0    -2.866      2.598
a3-a4        10874.0   21939.0   15360.0    1.0    -2.786      2.702
μ1−μ2−μ3+μ4  10512.0   27357.0   14806.0    1.0    -5.659      5.032
\end{Verbatim}
\end{tcolorbox}
        
    \begin{tcolorbox}[breakable, size=fbox, boxrule=1pt, pad at break*=1mm,colback=cellbackground, colframe=cellborder]
\prompt{In}{incolor}{43}{\boxspacing}
\begin{Verbatim}[commandchars=\\\{\}]
\PY{n+nb}{print}\PY{p}{(}\PY{l+s+s2}{\PYZdq{}}\PY{l+s+s2}{95}\PY{l+s+si}{\PYZpc{} c}\PY{l+s+s2}{redible set of μ1−μ2−μ3+μ4 : [}\PY{l+s+si}{\PYZob{}\PYZcb{}}\PY{l+s+s2}{ , }\PY{l+s+si}{\PYZob{}\PYZcb{}}\PY{l+s+s2}{]}\PY{l+s+s2}{\PYZdq{}}\PY{o}{.}\PY{n}{format}\PY{p}{(}\PY{o}{\PYZhy{}}\PY{l+m+mf}{5.659}\PY{p}{,} \PY{l+m+mf}{5.032}\PY{p}{)} \PY{p}{)}
\end{Verbatim}
\end{tcolorbox}

    \begin{Verbatim}[commandchars=\\\{\}]
95\% credible set of μ1−μ2−μ3+μ4 : [-5.659 , 5.032]
    \end{Verbatim}

    This scenario is to test \(H_0 : \mu_1 + \mu_4 = \mu_3 + \mu_2 = 0\) VS.
\(H_1 : \mu_1 + \mu_4 \neq \mu_2 + \mu_3\)

\((\mu_1+\mu_4) - (\mu_2+\mu_3)\) still covers 0. So two treatments
which are added together more likely do not show difference of heights.

    \hypertarget{hockingpendleton-data}{%
\section{Hocking--Pendleton Data}\label{hockingpendleton-data}}

This popular data set was constructed by Hocking and Pendelton (1982) to
illustrate influential and outlier observations in regression. The data
are organized as a matrix of size 26 × 4; the predictors \(x_1\) ,
\(x_2\) , and \(x_3\) are the first three columns, and the response y is
the fourth column. The data are given in hockpend.dat

    \begin{tcolorbox}[breakable, size=fbox, boxrule=1pt, pad at break*=1mm,colback=cellbackground, colframe=cellborder]
\prompt{In}{incolor}{22}{\boxspacing}
\begin{Verbatim}[commandchars=\\\{\}]
\PY{n}{df3} \PY{o}{=} \PY{n}{pd}\PY{o}{.}\PY{n}{read\PYZus{}csv}\PY{p}{(}\PY{l+s+s2}{\PYZdq{}}\PY{l+s+s2}{hockpend.dat}\PY{l+s+s2}{\PYZdq{}}\PY{p}{,} \PY{n}{sep}\PY{o}{=}\PY{l+s+s2}{\PYZdq{}}\PY{l+s+se}{\PYZbs{}t}\PY{l+s+s2}{\PYZdq{}}\PY{p}{,} \PY{n}{header}\PY{o}{=}\PY{k+kc}{None}\PY{p}{)}\PY{o}{.}\PY{n}{rename}\PY{p}{(}\PY{n}{columns}\PY{o}{=}\PY{p}{\PYZob{}}\PY{l+m+mi}{0}\PY{p}{:}\PY{l+s+s2}{\PYZdq{}}\PY{l+s+s2}{x1}\PY{l+s+s2}{\PYZdq{}}\PY{p}{,} \PY{l+m+mi}{1}\PY{p}{:}\PY{l+s+s2}{\PYZdq{}}\PY{l+s+s2}{x2}\PY{l+s+s2}{\PYZdq{}}\PY{p}{,} \PY{l+m+mi}{2}\PY{p}{:}\PY{l+s+s2}{\PYZdq{}}\PY{l+s+s2}{x3}\PY{l+s+s2}{\PYZdq{}}\PY{p}{,} \PY{l+m+mi}{3}\PY{p}{:} \PY{l+s+s2}{\PYZdq{}}\PY{l+s+s2}{y}\PY{l+s+s2}{\PYZdq{}}\PY{p}{\PYZcb{}}\PY{p}{)}
\end{Verbatim}
\end{tcolorbox}

    \begin{tcolorbox}[breakable, size=fbox, boxrule=1pt, pad at break*=1mm,colback=cellbackground, colframe=cellborder]
\prompt{In}{incolor}{23}{\boxspacing}
\begin{Verbatim}[commandchars=\\\{\}]
\PY{n}{df3}\PY{o}{.}\PY{n}{head}\PY{p}{(}\PY{p}{)}
\end{Verbatim}
\end{tcolorbox}

            \begin{tcolorbox}[breakable, size=fbox, boxrule=.5pt, pad at break*=1mm, opacityfill=0]
\prompt{Out}{outcolor}{23}{\boxspacing}
\begin{Verbatim}[commandchars=\\\{\}]
       x1     x2     x3       y
0  12.980  0.317  9.998  57.702
1  14.295  2.028  6.776  59.295
2  15.531  5.305  2.947  55.166
3  15.133  4.738  4.201  55.767
4  15.342  7.038  2.053  51.722
\end{Verbatim}
\end{tcolorbox}
        
    \hypertarget{a-fit-the-linear-regression-model-with-the-three-covariates-report-the-parameter-estimates-and-bayesian-r2}{%
\subsection{\texorpdfstring{(a) Fit the linear regression model with the
three covariates, report the parameter estimates and Bayesian
\(R^2\)}{(a) Fit the linear regression model with the three covariates, report the parameter estimates and Bayesian R\^{}2}}\label{a-fit-the-linear-regression-model-with-the-three-covariates-report-the-parameter-estimates-and-bayesian-r2}}

    \begin{tcolorbox}[breakable, size=fbox, boxrule=1pt, pad at break*=1mm,colback=cellbackground, colframe=cellborder]
\prompt{In}{incolor}{24}{\boxspacing}
\begin{Verbatim}[commandchars=\\\{\}]
\PY{k}{with} \PY{n}{pm}\PY{o}{.}\PY{n}{Model}\PY{p}{(}\PY{p}{)} \PY{k}{as} \PY{n}{linear\PYZus{}model}\PY{p}{:}
    \PY{n}{pm}\PY{o}{.}\PY{n}{glm}\PY{o}{.}\PY{n}{linear}\PY{o}{.}\PY{n}{GLM}\PY{p}{(}\PY{n}{y}\PY{o}{=}\PY{n}{df3}\PY{p}{[}\PY{l+s+s2}{\PYZdq{}}\PY{l+s+s2}{y}\PY{l+s+s2}{\PYZdq{}}\PY{p}{]}\PY{p}{,} \PY{n}{x}\PY{o}{=} \PY{n}{df3}\PY{p}{[}\PY{p}{[}\PY{l+s+s2}{\PYZdq{}}\PY{l+s+s2}{x1}\PY{l+s+s2}{\PYZdq{}}\PY{p}{,} \PY{l+s+s2}{\PYZdq{}}\PY{l+s+s2}{x2}\PY{l+s+s2}{\PYZdq{}}\PY{p}{,} \PY{l+s+s2}{\PYZdq{}}\PY{l+s+s2}{x3}\PY{l+s+s2}{\PYZdq{}}\PY{p}{]}\PY{p}{]}\PY{p}{,} \PY{n}{intercept}\PY{o}{=}\PY{k+kc}{True}\PY{p}{,} 
                      \PY{n}{family}\PY{o}{=}\PY{n}{pm}\PY{o}{.}\PY{n}{glm}\PY{o}{.}\PY{n}{families}\PY{o}{.}\PY{n}{Normal}\PY{p}{(}\PY{p}{)}\PY{p}{)}
    \PY{n}{trace\PYZus{}lm} \PY{o}{=} \PY{n}{pm}\PY{o}{.}\PY{n}{sample}\PY{p}{(}\PY{l+m+mi}{5000}\PY{p}{,} \PY{n}{tune}\PY{o}{=}\PY{l+m+mi}{5000}\PY{p}{,} \PY{n}{init}\PY{o}{=}\PY{l+s+s1}{\PYZsq{}}\PY{l+s+s1}{adapt\PYZus{}diag}\PY{l+s+s1}{\PYZsq{}}\PY{p}{)}
    \PY{n}{posterior\PYZus{}predictive\PYZus{}lm} \PY{o}{=} \PY{n}{pm}\PY{o}{.}\PY{n}{sample\PYZus{}posterior\PYZus{}predictive}\PY{p}{(}\PY{n}{trace\PYZus{}lm}\PY{p}{)}
\end{Verbatim}
\end{tcolorbox}

    \begin{Verbatim}[commandchars=\\\{\}]
Auto-assigning NUTS sampler{\ldots}
Initializing NUTS using adapt\_diag{\ldots}
Multiprocess sampling (4 chains in 4 jobs)
NUTS: [sd, x3, x2, x1, Intercept]
    \end{Verbatim}

    
    \begin{verbatim}
<IPython.core.display.HTML object>
    \end{verbatim}

    
    \begin{Verbatim}[commandchars=\\\{\}]
Sampling 4 chains for 5\_000 tune and 5\_000 draw iterations (20\_000 + 20\_000
draws total) took 90 seconds.
There were 249 divergences after tuning. Increase `target\_accept` or
reparameterize.
There were 125 divergences after tuning. Increase `target\_accept` or
reparameterize.
There were 1695 divergences after tuning. Increase `target\_accept` or
reparameterize.
The acceptance probability does not match the target. It is 0.4309795324404258,
but should be close to 0.8. Try to increase the number of tuning steps.
There were 26 divergences after tuning. Increase `target\_accept` or
reparameterize.
The rhat statistic is larger than 1.05 for some parameters. This indicates
slight problems during sampling.
The estimated number of effective samples is smaller than 200 for some
parameters.
    \end{Verbatim}

    
    \begin{verbatim}
<IPython.core.display.HTML object>
    \end{verbatim}

    
    \begin{tcolorbox}[breakable, size=fbox, boxrule=1pt, pad at break*=1mm,colback=cellbackground, colframe=cellborder]
\prompt{In}{incolor}{25}{\boxspacing}
\begin{Verbatim}[commandchars=\\\{\}]
\PY{n}{az}\PY{o}{.}\PY{n}{summary}\PY{p}{(}\PY{n}{trace\PYZus{}lm}\PY{p}{)}
\end{Verbatim}
\end{tcolorbox}

    \begin{Verbatim}[commandchars=\\\{\}]
/opt/anaconda3/lib/python3.7/site-packages/arviz/data/io\_pymc3.py:91:
FutureWarning: Using `from\_pymc3` without the model will be deprecated in a
future release. Not using the model will return less accurate and less useful
results. Make sure you use the model argument or call from\_pymc3 within a model
context.
  FutureWarning,
    \end{Verbatim}

            \begin{tcolorbox}[breakable, size=fbox, boxrule=.5pt, pad at break*=1mm, opacityfill=0]
\prompt{Out}{outcolor}{25}{\boxspacing}
\begin{Verbatim}[commandchars=\\\{\}]
            mean     sd  hdi\_3\%  hdi\_97\%  mcse\_mean  mcse\_sd  ess\_mean  \textbackslash{}
Intercept  9.303  9.702  -9.351   25.252      1.588    1.132      37.0
x1         3.399  0.547   2.505    4.487      0.085    0.060      42.0
x2        -1.461  0.251  -1.874   -0.972      0.027    0.019      86.0
x3         0.318  0.280  -0.162    0.840      0.040    0.029      48.0
sd         2.638  0.438   1.847    3.406      0.043    0.031     102.0

           ess\_sd  ess\_bulk  ess\_tail  r\_hat
Intercept    37.0      37.0     784.0   1.07
x1           42.0      41.0    1411.0   1.06
x2           86.0      86.0     745.0   1.03
x3           48.0      47.0     644.0   1.06
sd          102.0      76.0      47.0   1.03
\end{Verbatim}
\end{tcolorbox}
        
    There are five estimated parameters. The estimated coefficients of the
intercept, \(x_1\), \(x_2\), \(x_3\) and standard deviation are 9.303,
3.399, -1.461, 0.318 and 2.638, respectively.

    \begin{tcolorbox}[breakable, size=fbox, boxrule=1pt, pad at break*=1mm,colback=cellbackground, colframe=cellborder]
\prompt{In}{incolor}{26}{\boxspacing}
\begin{Verbatim}[commandchars=\\\{\}]
\PY{n}{r2\PYZus{}scores} \PY{o}{=} \PY{p}{[}\PY{p}{]}
\PY{n}{y\PYZus{}true} \PY{o}{=} \PY{n}{df3}\PY{p}{[}\PY{l+s+s2}{\PYZdq{}}\PY{l+s+s2}{y}\PY{l+s+s2}{\PYZdq{}}\PY{p}{]}
\PY{k}{for} \PY{n}{i} \PY{o+ow}{in} \PY{n+nb}{range}\PY{p}{(}\PY{n+nb}{len}\PY{p}{(}\PY{n}{posterior\PYZus{}predictive\PYZus{}lm}\PY{p}{[}\PY{l+s+s2}{\PYZdq{}}\PY{l+s+s2}{y}\PY{l+s+s2}{\PYZdq{}}\PY{p}{]}\PY{p}{)}\PY{p}{)}\PY{p}{:}
    \PY{n}{y\PYZus{}pred} \PY{o}{=} \PY{n}{posterior\PYZus{}predictive\PYZus{}lm}\PY{p}{[}\PY{l+s+s2}{\PYZdq{}}\PY{l+s+s2}{y}\PY{l+s+s2}{\PYZdq{}}\PY{p}{]}\PY{p}{[}\PY{n}{i}\PY{p}{]}
    \PY{n}{r2} \PY{o}{=} \PY{n}{az}\PY{o}{.}\PY{n}{r2\PYZus{}score}\PY{p}{(}\PY{n}{y\PYZus{}true}\PY{p}{,} \PY{n}{y\PYZus{}pred}\PY{p}{)}\PY{p}{[}\PY{l+m+mi}{0}\PY{p}{]}
    \PY{n}{r2\PYZus{}scores}\PY{o}{.}\PY{n}{append}\PY{p}{(}\PY{n}{r2}\PY{p}{)}
\PY{n+nb}{print}\PY{p}{(}\PY{l+s+s2}{\PYZdq{}}\PY{l+s+s2}{Mean of BR2: }\PY{l+s+s2}{\PYZdq{}}\PY{p}{,} \PY{n}{np}\PY{o}{.}\PY{n}{mean}\PY{p}{(}\PY{n}{r2\PYZus{}scores}\PY{p}{)}\PY{p}{)}
\PY{n+nb}{print}\PY{p}{(}\PY{l+s+s2}{\PYZdq{}}\PY{l+s+s2}{Standard deviation of BR2: }\PY{l+s+s2}{\PYZdq{}}\PY{p}{,} \PY{n}{np}\PY{o}{.}\PY{n}{std}\PY{p}{(}\PY{n}{r2\PYZus{}scores}\PY{p}{)}\PY{p}{)}
\end{Verbatim}
\end{tcolorbox}

    \begin{Verbatim}[commandchars=\\\{\}]
Mean of BR2:  0.7605379992456223
Standard deviation of BR2:  0.061840177107345136
    \end{Verbatim}

    \hypertarget{b-is-any-of-the-26-observations-influential-or-outlier-in-the-sense-of-cpo-and-cumulative}{%
\subsection{(b) Is any of the 26 observations influential or outlier (in
the sense of CPO and
cumulative)?}\label{b-is-any-of-the-26-observations-influential-or-outlier-in-the-sense-of-cpo-and-cumulative}}

    \hypertarget{cumulative}{%
\subsubsection{Cumulative}\label{cumulative}}

    We use Cumulative to check whether each data point is an outlier. The
concpt is to use the distribution defined in each iteration and then we
will check where the observed value locates in its cumulative
distribution. So after simulation, we could see the means of each data
point. If the data point is close to 1 or 0, it is more likely to
conclude it is an outlier.

Based on the above samples, our model is:

\(y_i = \beta_0 + \beta_1 * x_{i1} + \beta_2 * x_{i2} + \beta_3 * x_{i3} + \epsilon_i,\ where\ \epsilon_i \sim N(0, \sigma^2),\ i = 1, 2, \dots, 26\)

And, we define each coefficient except \(\beta_0\) as \(N(0, 10^{-5})\).
As for the intercept, we use the Flat prior.

\(y_i \sim N(\beta_0 + \beta_1 * x_{i1} + \beta_2 * x_{i2} + \beta_3 * x_{i3}, \eta=sd^{2})\)

    \begin{tcolorbox}[breakable, size=fbox, boxrule=1pt, pad at break*=1mm,colback=cellbackground, colframe=cellborder]
\prompt{In}{incolor}{27}{\boxspacing}
\begin{Verbatim}[commandchars=\\\{\}]
\PY{k+kn}{from} \PY{n+nn}{scipy}\PY{n+nn}{.}\PY{n+nn}{stats} \PY{k+kn}{import} \PY{n}{norm}
\end{Verbatim}
\end{tcolorbox}

    \begin{tcolorbox}[breakable, size=fbox, boxrule=1pt, pad at break*=1mm,colback=cellbackground, colframe=cellborder]
\prompt{In}{incolor}{28}{\boxspacing}
\begin{Verbatim}[commandchars=\\\{\}]
\PY{n}{df\PYZus{}trace} \PY{o}{=} \PY{n}{pm}\PY{o}{.}\PY{n}{backends}\PY{o}{.}\PY{n}{tracetab}\PY{o}{.}\PY{n}{trace\PYZus{}to\PYZus{}dataframe}\PY{p}{(}\PY{n}{trace\PYZus{}lm}\PY{p}{)}
\PY{n}{df\PYZus{}trace}\PY{o}{.}\PY{n}{head}\PY{p}{(}\PY{p}{)}
\end{Verbatim}
\end{tcolorbox}

            \begin{tcolorbox}[breakable, size=fbox, boxrule=.5pt, pad at break*=1mm, opacityfill=0]
\prompt{Out}{outcolor}{28}{\boxspacing}
\begin{Verbatim}[commandchars=\\\{\}]
   Intercept        x1        x2        x3        sd
0   5.601625  3.622686 -1.233575  0.274603  2.855812
1   7.631317  3.312460 -1.100052  0.625191  2.919152
2   5.730193  3.375129 -0.908699  0.620648  2.695725
3   5.979099  3.553475 -1.339585  0.523389  3.046800
4   3.279882  3.805025 -1.496413  0.331836  2.304360
\end{Verbatim}
\end{tcolorbox}
        
    \begin{tcolorbox}[breakable, size=fbox, boxrule=1pt, pad at break*=1mm,colback=cellbackground, colframe=cellborder]
\prompt{In}{incolor}{29}{\boxspacing}
\begin{Verbatim}[commandchars=\\\{\}]
\PY{n}{cuy} \PY{o}{=} \PY{n}{np}\PY{o}{.}\PY{n}{zeros}\PY{p}{(}\PY{l+m+mi}{26}\PY{p}{)}

\PY{k}{for} \PY{n}{i} \PY{o+ow}{in} \PY{n+nb}{range}\PY{p}{(}\PY{n}{df\PYZus{}trace}\PY{o}{.}\PY{n}{shape}\PY{p}{[}\PY{l+m+mi}{0}\PY{p}{]}\PY{p}{)}\PY{p}{:}
    \PY{n}{intercept} \PY{o}{=} \PY{n}{df\PYZus{}trace}\PY{o}{.}\PY{n}{iloc}\PY{p}{[}\PY{n}{i}\PY{p}{,} \PY{l+m+mi}{0}\PY{p}{]}
    \PY{n}{b1} \PY{o}{=} \PY{n}{df\PYZus{}trace}\PY{o}{.}\PY{n}{iloc}\PY{p}{[}\PY{n}{i}\PY{p}{,} \PY{l+m+mi}{1}\PY{p}{]}
    \PY{n}{b2} \PY{o}{=} \PY{n}{df\PYZus{}trace}\PY{o}{.}\PY{n}{iloc}\PY{p}{[}\PY{n}{i}\PY{p}{,} \PY{l+m+mi}{2}\PY{p}{]}
    \PY{n}{b3} \PY{o}{=} \PY{n}{df\PYZus{}trace}\PY{o}{.}\PY{n}{iloc}\PY{p}{[}\PY{n}{i}\PY{p}{,} \PY{l+m+mi}{3}\PY{p}{]}
    \PY{n}{sd} \PY{o}{=} \PY{n}{df\PYZus{}trace}\PY{o}{.}\PY{n}{iloc}\PY{p}{[}\PY{n}{i}\PY{p}{,} \PY{l+m+mi}{4}\PY{p}{]}
    \PY{k}{for} \PY{n}{j} \PY{o+ow}{in} \PY{n+nb}{range}\PY{p}{(}\PY{l+m+mi}{26}\PY{p}{)}\PY{p}{:}
        \PY{n}{obs} \PY{o}{=} \PY{n}{df3}\PY{p}{[}\PY{l+s+s2}{\PYZdq{}}\PY{l+s+s2}{y}\PY{l+s+s2}{\PYZdq{}}\PY{p}{]}\PY{p}{[}\PY{n}{j}\PY{p}{]}
        \PY{n}{cuy}\PY{p}{[}\PY{n}{j}\PY{p}{]} \PY{o}{+}\PY{o}{=} \PY{n}{norm}\PY{o}{.}\PY{n}{cdf}\PY{p}{(}\PY{n}{obs}\PY{p}{,} \PY{n}{loc}\PY{o}{=}\PY{n}{intercept} \PY{o}{+} \PY{n}{b1} \PY{o}{*} \PY{n}{df3}\PY{p}{[}\PY{l+s+s2}{\PYZdq{}}\PY{l+s+s2}{x1}\PY{l+s+s2}{\PYZdq{}}\PY{p}{]}\PY{p}{[}\PY{n}{j}\PY{p}{]} \PY{o}{+} \PY{n}{b2} \PY{o}{*} \PY{n}{df3}\PY{p}{[}\PY{l+s+s2}{\PYZdq{}}\PY{l+s+s2}{x2}\PY{l+s+s2}{\PYZdq{}}\PY{p}{]}\PY{p}{[}\PY{n}{j}\PY{p}{]} \PY{o}{+} \PY{n}{b3} \PY{o}{*} \PY{n}{df3}\PY{p}{[}\PY{l+s+s2}{\PYZdq{}}\PY{l+s+s2}{x3}\PY{l+s+s2}{\PYZdq{}}\PY{p}{]}\PY{p}{[}\PY{n}{j}\PY{p}{]}\PY{p}{,} \PY{n}{scale}\PY{o}{=}\PY{n}{sd}\PY{p}{)} 
\end{Verbatim}
\end{tcolorbox}

    \begin{tcolorbox}[breakable, size=fbox, boxrule=1pt, pad at break*=1mm,colback=cellbackground, colframe=cellborder]
\prompt{In}{incolor}{30}{\boxspacing}
\begin{Verbatim}[commandchars=\\\{\}]
\PY{n}{outlier\PYZus{}check} \PY{o}{=} \PY{n}{cuy} \PY{o}{/} \PY{n}{df\PYZus{}trace}\PY{o}{.}\PY{n}{shape}\PY{p}{[}\PY{l+m+mi}{0}\PY{p}{]}
\end{Verbatim}
\end{tcolorbox}

    \begin{tcolorbox}[breakable, size=fbox, boxrule=1pt, pad at break*=1mm,colback=cellbackground, colframe=cellborder]
\prompt{In}{incolor}{31}{\boxspacing}
\begin{Verbatim}[commandchars=\\\{\}]
\PY{n}{outlier\PYZus{}check}
\end{Verbatim}
\end{tcolorbox}

            \begin{tcolorbox}[breakable, size=fbox, boxrule=.5pt, pad at break*=1mm, opacityfill=0]
\prompt{Out}{outcolor}{31}{\boxspacing}
\begin{Verbatim}[commandchars=\\\{\}]
array([0.70828015, 0.79224846, 0.48248859, 0.5920942 , 0.4845708 ,
       0.72007163, 0.69536282, 0.39574399, 0.66401189, 0.69294881,
       0.22093436, 0.52572268, 0.62230292, 0.3701755 , 0.00388224,
       0.36372404, 0.94331224, 0.03018092, 0.30486989, 0.77054044,
       0.53216637, 0.68297244, 0.56086072, 0.5843513 , 0.47637694,
       0.40871561])
\end{Verbatim}
\end{tcolorbox}
        
    \begin{tcolorbox}[breakable, size=fbox, boxrule=1pt, pad at break*=1mm,colback=cellbackground, colframe=cellborder]
\prompt{In}{incolor}{32}{\boxspacing}
\begin{Verbatim}[commandchars=\\\{\}]
\PY{n}{outlier\PYZus{}check}\PY{p}{[}\PY{n}{outlier\PYZus{}check}\PY{o}{\PYZlt{}}\PY{l+m+mf}{0.1}\PY{p}{]}
\end{Verbatim}
\end{tcolorbox}

            \begin{tcolorbox}[breakable, size=fbox, boxrule=.5pt, pad at break*=1mm, opacityfill=0]
\prompt{Out}{outcolor}{32}{\boxspacing}
\begin{Verbatim}[commandchars=\\\{\}]
array([0.00388224, 0.03018092])
\end{Verbatim}
\end{tcolorbox}
        
    \begin{tcolorbox}[breakable, size=fbox, boxrule=1pt, pad at break*=1mm,colback=cellbackground, colframe=cellborder]
\prompt{In}{incolor}{33}{\boxspacing}
\begin{Verbatim}[commandchars=\\\{\}]
\PY{n}{outlier\PYZus{}check}\PY{p}{[}\PY{n}{outlier\PYZus{}check}\PY{o}{\PYZgt{}}\PY{l+m+mf}{0.9}\PY{p}{]}
\end{Verbatim}
\end{tcolorbox}

            \begin{tcolorbox}[breakable, size=fbox, boxrule=.5pt, pad at break*=1mm, opacityfill=0]
\prompt{Out}{outcolor}{33}{\boxspacing}
\begin{Verbatim}[commandchars=\\\{\}]
array([0.94331224])
\end{Verbatim}
\end{tcolorbox}
        
    \begin{tcolorbox}[breakable, size=fbox, boxrule=1pt, pad at break*=1mm,colback=cellbackground, colframe=cellborder]
\prompt{In}{incolor}{45}{\boxspacing}
\begin{Verbatim}[commandchars=\\\{\}]
\PY{n}{np}\PY{o}{.}\PY{n}{where}\PY{p}{(}\PY{n}{outlier\PYZus{}check}\PY{o}{==}\PY{l+m+mf}{0.0038822393888119306}\PY{p}{)}\PY{p}{[}\PY{l+m+mi}{0}\PY{p}{]}\PY{o}{+}\PY{l+m+mi}{1}
\end{Verbatim}
\end{tcolorbox}

            \begin{tcolorbox}[breakable, size=fbox, boxrule=.5pt, pad at break*=1mm, opacityfill=0]
\prompt{Out}{outcolor}{45}{\boxspacing}
\begin{Verbatim}[commandchars=\\\{\}]
array([15])
\end{Verbatim}
\end{tcolorbox}
        
    \begin{tcolorbox}[breakable, size=fbox, boxrule=1pt, pad at break*=1mm,colback=cellbackground, colframe=cellborder]
\prompt{In}{incolor}{46}{\boxspacing}
\begin{Verbatim}[commandchars=\\\{\}]
\PY{n}{np}\PY{o}{.}\PY{n}{where}\PY{p}{(}\PY{n}{outlier\PYZus{}check}\PY{o}{==} \PY{l+m+mf}{0.030180919979533604}\PY{p}{)}\PY{p}{[}\PY{l+m+mi}{0}\PY{p}{]}\PY{o}{+}\PY{l+m+mi}{1}
\end{Verbatim}
\end{tcolorbox}

            \begin{tcolorbox}[breakable, size=fbox, boxrule=.5pt, pad at break*=1mm, opacityfill=0]
\prompt{Out}{outcolor}{46}{\boxspacing}
\begin{Verbatim}[commandchars=\\\{\}]
array([18])
\end{Verbatim}
\end{tcolorbox}
        
    \begin{tcolorbox}[breakable, size=fbox, boxrule=1pt, pad at break*=1mm,colback=cellbackground, colframe=cellborder]
\prompt{In}{incolor}{47}{\boxspacing}
\begin{Verbatim}[commandchars=\\\{\}]
\PY{n}{np}\PY{o}{.}\PY{n}{where}\PY{p}{(}\PY{n}{outlier\PYZus{}check}\PY{o}{==}\PY{l+m+mf}{0.9433122366137262}\PY{p}{)}\PY{p}{[}\PY{l+m+mi}{0}\PY{p}{]}\PY{o}{+}\PY{l+m+mi}{1}
\end{Verbatim}
\end{tcolorbox}

            \begin{tcolorbox}[breakable, size=fbox, boxrule=.5pt, pad at break*=1mm, opacityfill=0]
\prompt{Out}{outcolor}{47}{\boxspacing}
\begin{Verbatim}[commandchars=\\\{\}]
array([17])
\end{Verbatim}
\end{tcolorbox}
        
    Originally, I chose the \(\alpha = 0.2\) and found the 15th, 17th and
18th observations are more likely outliers since the means of these
observations are either too close to 0 and 1.

If we choose the smaller confidence level, like we change \(\alpha\) to
\(\alpha=0.05\), the obvious outlier is the 15th data point.

    \hypertarget{c-find-the-mean-response-and-prediction-response-for-a-new-observation-with-covariates-x_1-10-x_2-5-and-x_3-5.-report-the-corresponding-95-credible-sets}{%
\subsection{\texorpdfstring{(c) Find the mean response and prediction
response for a new observation with covariates \(x^{*}_1 = 10\),
\(x^{*}_2 = 5\), and \(x^{*}_3 = 5\). Report the corresponding 95\%
credible
sets}{(c) Find the mean response and prediction response for a new observation with covariates x\^{}\{*\}\_1 = 10, x\^{}\{*\}\_2 = 5, and x\^{}\{*\}\_3 = 5. Report the corresponding 95\% credible sets}}\label{c-find-the-mean-response-and-prediction-response-for-a-new-observation-with-covariates-x_1-10-x_2-5-and-x_3-5.-report-the-corresponding-95-credible-sets}}

    \begin{tcolorbox}[breakable, size=fbox, boxrule=1pt, pad at break*=1mm,colback=cellbackground, colframe=cellborder]
\prompt{In}{incolor}{37}{\boxspacing}
\begin{Verbatim}[commandchars=\\\{\}]
\PY{n}{new\PYZus{}df} \PY{o}{=} \PY{n}{pd}\PY{o}{.}\PY{n}{DataFrame}\PY{p}{(}\PY{p}{\PYZob{}}\PY{l+s+s2}{\PYZdq{}}\PY{l+s+s2}{x1}\PY{l+s+s2}{\PYZdq{}}\PY{p}{:}\PY{p}{[}\PY{l+m+mi}{10}\PY{p}{]}\PY{p}{,} \PY{l+s+s2}{\PYZdq{}}\PY{l+s+s2}{x2}\PY{l+s+s2}{\PYZdq{}}\PY{p}{:} \PY{p}{[}\PY{l+m+mi}{5}\PY{p}{]}\PY{p}{,} \PY{l+s+s2}{\PYZdq{}}\PY{l+s+s2}{x3}\PY{l+s+s2}{\PYZdq{}}\PY{p}{:} \PY{p}{[}\PY{l+m+mi}{5}\PY{p}{]}\PY{p}{,} \PY{l+s+s2}{\PYZdq{}}\PY{l+s+s2}{y}\PY{l+s+s2}{\PYZdq{}}\PY{p}{:}\PY{p}{[}\PY{l+m+mf}{4.813} \PY{o}{+} \PY{l+m+mf}{3.648} \PY{o}{*} \PY{l+m+mi}{10} \PY{o}{+} \PY{p}{(}\PY{o}{\PYZhy{}}\PY{l+m+mf}{1.400}\PY{p}{)} \PY{o}{*} \PY{l+m+mi}{5} \PY{o}{+} \PY{l+m+mf}{0.420} \PY{o}{*} \PY{l+m+mi}{5}\PY{p}{]}\PY{p}{\PYZcb{}}\PY{p}{)}
\end{Verbatim}
\end{tcolorbox}

    \begin{tcolorbox}[breakable, size=fbox, boxrule=1pt, pad at break*=1mm,colback=cellbackground, colframe=cellborder]
\prompt{In}{incolor}{38}{\boxspacing}
\begin{Verbatim}[commandchars=\\\{\}]
\PY{k}{with} \PY{n}{pm}\PY{o}{.}\PY{n}{Model}\PY{p}{(}\PY{p}{)} \PY{k}{as} \PY{n}{lm\PYZus{}pred}\PY{p}{:}    
    \PY{n}{pm}\PY{o}{.}\PY{n}{glm}\PY{o}{.}\PY{n}{linear}\PY{o}{.}\PY{n}{GLM}\PY{p}{(}\PY{n}{y}\PY{o}{=}\PY{n}{new\PYZus{}df}\PY{p}{[}\PY{l+s+s2}{\PYZdq{}}\PY{l+s+s2}{y}\PY{l+s+s2}{\PYZdq{}}\PY{p}{]}\PY{p}{,}
                      \PY{n}{x}\PY{o}{=}\PY{n}{new\PYZus{}df}\PY{p}{[}\PY{p}{[}\PY{l+s+s2}{\PYZdq{}}\PY{l+s+s2}{x1}\PY{l+s+s2}{\PYZdq{}}\PY{p}{,} \PY{l+s+s2}{\PYZdq{}}\PY{l+s+s2}{x2}\PY{l+s+s2}{\PYZdq{}}\PY{p}{,} \PY{l+s+s2}{\PYZdq{}}\PY{l+s+s2}{x3}\PY{l+s+s2}{\PYZdq{}}\PY{p}{]}\PY{p}{]}\PY{p}{,} \PY{n}{intercept}\PY{o}{=}\PY{k+kc}{True}\PY{p}{,} 
                      \PY{n}{priors}\PY{o}{=}\PY{p}{\PYZob{}}\PY{l+s+s1}{\PYZsq{}}\PY{l+s+s1}{Intercept}\PY{l+s+s1}{\PYZsq{}}\PY{p}{:}\PY{n}{pm}\PY{o}{.}\PY{n}{Normal}\PY{o}{.}\PY{n}{dist}\PY{p}{(}\PY{n}{mu}\PY{o}{=}\PY{l+m+mi}{0}\PY{p}{,} \PY{n}{sigma}\PY{o}{=}\PY{l+m+mf}{10.0}\PY{p}{)}\PY{p}{\PYZcb{}}\PY{p}{,}
                      \PY{n}{family}\PY{o}{=}\PY{n}{pm}\PY{o}{.}\PY{n}{glm}\PY{o}{.}\PY{n}{families}\PY{o}{.}\PY{n}{Normal}\PY{p}{(}\PY{p}{)}\PY{p}{)}
    \PY{n}{trace\PYZus{}lm\PYZus{}p} \PY{o}{=} \PY{n}{pm}\PY{o}{.}\PY{n}{sample}\PY{p}{(}\PY{l+m+mi}{5000}\PY{p}{,} \PY{n}{tune}\PY{o}{=}\PY{l+m+mi}{5000}\PY{p}{,} \PY{n}{init}\PY{o}{=}\PY{l+s+s1}{\PYZsq{}}\PY{l+s+s1}{adapt\PYZus{}diag}\PY{l+s+s1}{\PYZsq{}}\PY{p}{)}
    \PY{n}{df} \PY{o}{=} \PY{n}{pm}\PY{o}{.}\PY{n}{trace\PYZus{}to\PYZus{}dataframe}\PY{p}{(}\PY{n}{trace\PYZus{}lm}\PY{p}{,} \PY{n}{include\PYZus{}transformed}\PY{o}{=}\PY{k+kc}{True}\PY{p}{)}
    \PY{n}{ppc} \PY{o}{=} \PY{n}{pm}\PY{o}{.}\PY{n}{sample\PYZus{}posterior\PYZus{}predictive}\PY{p}{(}\PY{n}{trace}\PY{o}{=}\PY{n}{df}\PY{o}{.}\PY{n}{to\PYZus{}dict}\PY{p}{(}\PY{l+s+s1}{\PYZsq{}}\PY{l+s+s1}{records}\PY{l+s+s1}{\PYZsq{}}\PY{p}{)}\PY{p}{,}\PY{n}{samples}\PY{o}{=}\PY{n+nb}{len}\PY{p}{(}\PY{n}{df}\PY{p}{)}\PY{p}{)}
\end{Verbatim}
\end{tcolorbox}

    \begin{Verbatim}[commandchars=\\\{\}]
Auto-assigning NUTS sampler{\ldots}
Initializing NUTS using adapt\_diag{\ldots}
Multiprocess sampling (4 chains in 4 jobs)
NUTS: [sd, x3, x2, x1, Intercept]
    \end{Verbatim}

    
    \begin{verbatim}
<IPython.core.display.HTML object>
    \end{verbatim}

    
    \begin{Verbatim}[commandchars=\\\{\}]
Sampling 4 chains for 5\_000 tune and 5\_000 draw iterations (20\_000 + 20\_000
draws total) took 39 seconds.
There were 2345 divergences after tuning. Increase `target\_accept` or
reparameterize.
The acceptance probability does not match the target. It is 0.3403953364767104,
but should be close to 0.8. Try to increase the number of tuning steps.
There were 1798 divergences after tuning. Increase `target\_accept` or
reparameterize.
The acceptance probability does not match the target. It is 0.44437999849911075,
but should be close to 0.8. Try to increase the number of tuning steps.
There were 3543 divergences after tuning. Increase `target\_accept` or
reparameterize.
The acceptance probability does not match the target. It is 0.05906703653418209,
but should be close to 0.8. Try to increase the number of tuning steps.
There were 2760 divergences after tuning. Increase `target\_accept` or
reparameterize.
The acceptance probability does not match the target. It is 0.30560815998471047,
but should be close to 0.8. Try to increase the number of tuning steps.
The rhat statistic is larger than 1.4 for some parameters. The sampler did not
converge.
The estimated number of effective samples is smaller than 200 for some
parameters.
    \end{Verbatim}

    
    \begin{verbatim}
<IPython.core.display.HTML object>
    \end{verbatim}

    
    Prediction response:

This means the result is derived based on inputs and coefficients.

    \begin{tcolorbox}[breakable, size=fbox, boxrule=1pt, pad at break*=1mm,colback=cellbackground, colframe=cellborder]
\prompt{In}{incolor}{1}{\boxspacing}
\begin{Verbatim}[commandchars=\\\{\}]
\PY{l+m+mf}{4.813} \PY{o}{+} \PY{l+m+mf}{3.648} \PY{o}{*} \PY{l+m+mi}{10} \PY{o}{+} \PY{p}{(}\PY{o}{\PYZhy{}}\PY{l+m+mf}{1.400}\PY{p}{)} \PY{o}{*} \PY{l+m+mi}{5} \PY{o}{+} \PY{l+m+mf}{0.420} \PY{o}{*} \PY{l+m+mi}{5}
\end{Verbatim}
\end{tcolorbox}

            \begin{tcolorbox}[breakable, size=fbox, boxrule=.5pt, pad at break*=1mm, opacityfill=0]
\prompt{Out}{outcolor}{1}{\boxspacing}
\begin{Verbatim}[commandchars=\\\{\}]
36.39300000000001
\end{Verbatim}
\end{tcolorbox}
        
    Mean response:

This means that based on each iteration, we get estimated coefficients
and then have multiple response values. We take expectation on these
values.

    \begin{tcolorbox}[breakable, size=fbox, boxrule=1pt, pad at break*=1mm,colback=cellbackground, colframe=cellborder]
\prompt{In}{incolor}{40}{\boxspacing}
\begin{Verbatim}[commandchars=\\\{\}]
\PY{n}{ppc}\PY{p}{[}\PY{l+s+s2}{\PYZdq{}}\PY{l+s+s2}{y}\PY{l+s+s2}{\PYZdq{}}\PY{p}{]}\PY{o}{.}\PY{n}{mean}\PY{p}{(}\PY{p}{)}
\end{Verbatim}
\end{tcolorbox}

            \begin{tcolorbox}[breakable, size=fbox, boxrule=.5pt, pad at break*=1mm, opacityfill=0]
\prompt{Out}{outcolor}{40}{\boxspacing}
\begin{Verbatim}[commandchars=\\\{\}]
37.576426132208155
\end{Verbatim}
\end{tcolorbox}
        
    \begin{tcolorbox}[breakable, size=fbox, boxrule=1pt, pad at break*=1mm,colback=cellbackground, colframe=cellborder]
\prompt{In}{incolor}{41}{\boxspacing}
\begin{Verbatim}[commandchars=\\\{\}]
\PY{n}{az}\PY{o}{.}\PY{n}{hdi}\PY{p}{(}\PY{n}{ppc}\PY{p}{[}\PY{l+s+s2}{\PYZdq{}}\PY{l+s+s2}{y}\PY{l+s+s2}{\PYZdq{}}\PY{p}{]}\PY{p}{,} \PY{n}{hdi\PYZus{}prob}\PY{o}{=}\PY{l+m+mf}{0.95}\PY{p}{)}
\end{Verbatim}
\end{tcolorbox}

    \begin{Verbatim}[commandchars=\\\{\}]
/opt/anaconda3/lib/python3.7/site-packages/arviz/stats/stats.py:487:
FutureWarning: hdi currently interprets 2d data as (draw, shape) but this will
change in a future release to (chain, draw) for coherence with other functions
  FutureWarning,
    \end{Verbatim}

            \begin{tcolorbox}[breakable, size=fbox, boxrule=.5pt, pad at break*=1mm, opacityfill=0]
\prompt{Out}{outcolor}{41}{\boxspacing}
\begin{Verbatim}[commandchars=\\\{\}]
array([[29.97973519, 44.50649613]])
\end{Verbatim}
\end{tcolorbox}
        
    The prediction response is 36.39300000000001.

The mean response is 37.576426132208155 and the 95\% credible set is
between 29.97973519 and 44.50649613.


    % Add a bibliography block to the postdoc
    
    
    
\end{document}
